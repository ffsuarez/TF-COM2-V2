En la Tabla \ref{tab:comp} se realiza una comparación entre las distintas ventajas y desventajas que presentan las dos propuestas de enlace troncal.

% Please add the following required packages to your document preamble:
% \usepackage[table,xcdraw]{xcolor}
% If you use beamer only pass "xcolor=table" option, i.e. \documentclass[xcolor=table]{beamer}
\begin{table}[H]
\centering
\scriptsize
\begin{tabular}{|
>{\columncolor[HTML]{C5D9F1}}l |l|l|}
\hline
\cellcolor[HTML]{C5D9F1}\textbf{Caracteristicas}                                     & \cellcolor[HTML]{C5D9F1}\textbf{Radio Enlace}                                                                                                                                                                                                                                                                                                                                                                             & \cellcolor[HTML]{C5D9F1}\textbf{Fibra Óptica}                                                                                                                                                                                                                                                                                                                                                                                                                                      \\ \hline
\textbf{Fiabilidad}                                                                  & \begin{tabular}[c]{@{}l@{}}Ventaja: El enlace presenta fiabilidad\\  debido al añadimiento de elementos\\  de reserva a través de un equipo\\  auxiliar ubicado sobre la estación\\  retransmisora en disposición\\  hot-stand by.\end{tabular}                                                                                                                                                                           & \begin{tabular}[c]{@{}l@{}}Ventaja: Al utilizarse un único pelo \\ de fibra, hay 5 restantes para utilizar\\  como remplazo frente al daño del\\  primero.\end{tabular}                                                                                                                                                                                                                                                                                                            \\ \hline
\textbf{\begin{tabular}[c]{@{}l@{}}Distancia del\\  enlace\end{tabular}}             & \begin{tabular}[c]{@{}l@{}}Desventaja: La distancia que debe \\ recorrerse entre estaciones es\\ considerable.\\ Esto se deriva en una complicación \\ frente a  fallas del enlace o en cuanto a tareas \\ de mantenimiento.\end{tabular} & \begin{tabular}[c]{@{}l@{}}Ventaja: El enlace posee una\\  longitud comparable a la distancia\\  en línea recta entre Papagayos y\\  Carpintería.\\  Esto presenta una ventaja en\\  cuanto a la distancia que se debe \\ recorrer frente a tareas de \\ mantenimiento y reparación \\ del enlace.\\ Facilita el acceso al tendido frente\\  a cualquier problemática o \\ mantención del mismo, ya que se \\ encuentra en las cercanías de la\\  Ruta Provincial N1.\end{tabular} \\ \hline
\textbf{\begin{tabular}[c]{@{}l@{}}Ubicación \\ de los \\ equipos\end{tabular}}     & \begin{tabular}[c]{@{}l@{}}Desventaja: Los equipos al estar \\ colocados a la intemperie pueden \\ sufrir daños.\end{tabular}                                                                                                                                                                                                                 & \begin{tabular}[c]{@{}l@{}}Ventaja: Los equipos se colocan\\  en el interior de una caseta por\\  lo que se encuentran protegidos.\end{tabular}                                                                                                                                                                                                                                                                                                                                    \\ \hline
\textbf{\begin{tabular}[c]{@{}l@{}}Capacidad de\\  futura \\ expansión\end{tabular}} & \begin{tabular}[c]{@{}l@{}}Desventaja: El repetidor activo \\ colocado en las cercanías de\\  Concarán no posee la capacidad\\  suficiente para abastecer a futuros\\  clientes pertenecientes al pueblo de\\  aproximadamente 5119 habitantes.\\  Solo una porción recibirían el \\ servicio.\end{tabular}                                                                                                               & \begin{tabular}[c]{@{}l@{}}Ventaja: El enlace tiene una gran \\ capacidad de transmisión, por lo\\  que ofrece sin ningún tipo de\\  dificultad una gran expansión de\\  prestación del servicio. La fibra\\  tiene 6 pelos y solo se ocupa 1,\\  los otros 5 se pueden\\  ocupar en los pueblos intermedios.\end{tabular}                                                                                                                                                         \\ \hline
\textbf{\begin{tabular}[c]{@{}l@{}}Despliegue de\\  la red\end{tabular}}             & \begin{tabular}[c]{@{}l@{}}Desventaja: La implementación de la\\  red se realiza  a través de instalación \\ de torres propias.\end{tabular}                                                                                                                                                                                                                                                                              & \begin{tabular}[c]{@{}l@{}}Ventaja: La implementación de\\  la red se realiza sobre el tendido\\  eléctrico perteneciente a EDESAL, \\ a lo largo de la Ruta Provincial N1,\\  lo que implica un gran ahorro\\  económico, debido a que el\\  alquiler de dichos \\ postes es considerablemente más \\ económico para el despliegue de\\  la red que la instalación de \\ postes propios.\end{tabular}                                                                             \\ \hline
\textbf{Precio Total}                                                                &  US\$                                                                                                                                                                                                                                                                                                                                                                                                   78.374,89 &  US\$ 81.199,62 \\ \hline
\end{tabular}
\caption{Comparacion Tecnologias para Enlace Troncal.}
\label{tab:comp}
\end{table}



Analizando ambas propuestas de establecimiento del enlace para cada ítem descrito, se opta por el enlace óptico, que a pesar de tener un 3,6\% de costo mayor respecto al radio enlace, presenta ser un enlace más prometedor con mayor cantidad  y calidad de ventajas.

 
 
 
 
 
 
 
 
 
 
 
 
 
 
 
 \subsection{Red de Acceso}
 

En la Tabla \ref{tab:APvsFO} se realiza una comparación entre las ventajas y desventajas que presenta cada uno de los enlaces frente a diversas características.

 
% Please add the following required packages to your document preamble:
% \usepackage[table,xcdraw]{xcolor}
% If you use beamer only pass "xcolor=table" option, i.e. \documentclass[xcolor=table]{beamer}
\begin{table}[H]
\scriptsize
\begin{tabular}{|
>{\columncolor[HTML]{C5D9F1}}c |c|c|}
\hline
{\color[HTML]{000000} \textbf{Características}} & \cellcolor[HTML]{C5D9F1}{\color[HTML]{333333} \textbf{APs}} & \cellcolor[HTML]{C5D9F1}{\color[HTML]{333333} \textbf{Fibra Óptica}} \\ \hline
{\color[HTML]{333333} \textbf{\begin{tabular}[c]{@{}c@{}}Alimentación\\ del Sistema\end{tabular}}} & \begin{tabular}[c]{@{}c@{}}Desventaja: Energía eléctrica\\ para AP's\end{tabular} & \begin{tabular}[c]{@{}c@{}}Ventaja: Los únicos elementos que\\ utilizan energía eléctrica\\ son los equipos terminales\end{tabular} \\ \hline
{\color[HTML]{333333} \textbf{Interferencia}} & \begin{tabular}[c]{@{}c@{}}Desventaja: Ruido e Interferencia\\ son elementos constantes\\ en este sistema\end{tabular} & \begin{tabular}[c]{@{}c@{}} Ventaja: Al ser una red óptica\\ no se presentan interferencias\\ electromagnéticas\end{tabular} \\ \hline
{\color[HTML]{333333} \textbf{Cobertura}} & Desventaja: Calculada por métodos empíricos & Ventaja: Asegurada completamente \\ \hline
{\color[HTML]{333333} \textbf{Tasas}} & \begin{tabular}[c]{@{}c@{}}Desventaja: La tasa de demanda inicial se encuentra asegurada.\\ Sin embargo esto no ocurre si se proyecta a futuro.\end{tabular} & Ventaja: Gran capacidad para cubrir la tasa inicial y futura. \\ \hline
{\color[HTML]{333333} \textbf{Mantenimiento}} & Ventaja: Son menos los elementos que requieren mantenimiento. & \begin{tabular}[c]{@{}c@{}} Desventaja: Se necesita un esfuerzo moderado:\\ Limpieza rutinaria\\ de conectores, patchera,\\ ordenamiento cables, revisión\\ de pérdidas, tensión del cable, estado de las fibras, entre\\ otros aspectos.\end{tabular} \\ \hline
{\color[HTML]{333333} \textbf{Infraestructura}} & Ventaja: Mayor Simpleza y Flexibilidad. & Desventaja: Cierto grado de rigidez estructural. \\ \hline
{\color[HTML]{333333} \textbf{Precio Total}} &US \$19.348,5  & US \$ 13.689,3 \\ \hline
\end{tabular}
\caption{Comparacion Tecnologias para Red de Acceso.}
\label{tab:APvsFO}
\end{table} 
 
 
 
 Analizando ambas propuestas  para cada ítem descrito, se opta por la red GPON, que a pesar de tener un 40\% de costo menor respecto a la red inalámbrica, presenta características mas atractivas, en especial se valora la capacidad de cobertura y expansión de la red que ofrece GPON.
 
%
 
\newpage
