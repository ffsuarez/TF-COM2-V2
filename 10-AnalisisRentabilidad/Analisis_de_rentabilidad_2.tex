\section{Análisis de rentabilidad}

La evaluación del proyecto es una operación que permite decidir si se realiza, o no, la inversión correspondiente para el desarrollo del mismo,en base a la comparación de los beneficios frente a los respectivos costos de inversión del proyecto.

Para esto, se procede al cálculo del valor Valor Actual Neto (VAN) y la Tasa Interna de Retorno (TIR), los cuales son indicadores para el cálculo de la viabilidad de un proyecto. Ambos conceptos se relacionan en forma directa con la estimación de flujo de fondo acumulado de la empresa y buscan hacer más preciso el cálculo del tiempo en que un negocio tardará en recuperar su inversión inicial.

La Figura \ref{dem-figuras:flujo_de_fondos} muestra la gráfica de flujo de fondo acumulado en un periodo de  años. Según dicha Figura, el proyecto comienza a tener un flujo positivo a partir del año 8.



% Table generated by Excel2LaTeX from sheet 'Hoja3'
%\begin{table}[H]
%  \small
%  \centering
%    \begin{tabular}{|c|c|}
%    \hline
%    \rowcolor[rgb]{ .773,  .851,  .945} \multicolumn{1}{|l|}{\textbf{Periodo}} & \multicolumn{1}{l|}{\textbf{Flujo de Fondos Operativos}} \bigstrut\\
%    \hline
%    \textbf{0} & -112.688,55 \bigstrut\\
%    \hline
%    \textbf{1} & 24.466,04 \bigstrut\\
%    \hline
%    \textbf{2} & 12.933,70 \bigstrut\\
%    \hline
%    \textbf{3} & 12.924,37 \bigstrut\\
%    \hline
%    \textbf{4} & 13.031,56 \bigstrut\\
%    \hline
%    \textbf{5} & 13.205,26 \bigstrut\\
%    \hline
%    \textbf{6} & 14.351,52 \bigstrut\\
%    \hline
%    \textbf{7} & 15.654,87 \bigstrut\\
%    \hline
%    \textbf{8} & 17.150,94 \bigstrut\\
%    \hline
%    \textbf{9} & 18.689,07 \bigstrut\\
%    \hline
%    \textbf{10} & 20.230,94 \bigstrut\\
%    \hline
%    \end{tabular}%
%     \caption{Flujo de Fondos Operativos.}
%  \label{tab:flujo-fondos-operativos}%
%\end{table}%


% Table generated by Excel2LaTeX from sheet 'Hoja2'
\begin{table}[H]
  \tiny
  \centering
    \begin{tabular}{|c|c|}
    \hline
    \rowcolor[rgb]{ .773,  .851,  .945} \textbf{Periodo} & \textbf{Flujo} \bigstrut\\
    \hline
    0     & -100952,48 \bigstrut\\
    \hline
    1     & -98852,26 \bigstrut\\
    \hline
    2     & -76952,46 \bigstrut\\
    \hline
    3     & -55081,85 \bigstrut\\
    \hline
    4     & -33095,66 \bigstrut\\
    \hline
    5     & -10938,37 \bigstrut\\
    \hline
    6     & 12146,86 \bigstrut\\
    \hline
    7     & 36314,16 \bigstrut\\
    \hline
    8     & 61755,02 \bigstrut\\
    \hline
    9     & 88510,72 \bigstrut\\
    \hline
    10    & 116584,94 \bigstrut\\
    \hline
    \end{tabular}%
     \caption{Flujo de Fondos Operativos.}
  \label{tab:flujo-fondos-operativos}%
\end{table}%




\FigureDEM{H}{8}{flujo_de_fondos}{Flujo de Fondos Acumulado.}

\subsection{VAN}

Pra que un negocio sea realmente rentable, el valor de la VAN debe ser siempre mayor que cero. Esto indica que en un plazo estimado, la inversión realizada para la marcha del proyecto, es recuperada y se obtendrá cierto beneficio. Se realiza el cálculo de la VAN en periodos de 4 y 6 años en la planilla de Excel adjunta.

\subsubsection{VAN a 4 años}

En la Tabla \ref{tab:van-4} se muestra el resultado del cálculo de la VAN en un periodo de 4 años para el flujo de fondo operativo indicado en la Tabla. Se utiliza una tasa de descuento del  1\%, tomada a la fecha del 4/3/2020, la cual pertenece a EEUU debido a que la cotización de todos los equipos y elementos para llevar a cabo el proyecto se realiza con la moneda dolar.


% Table generated by Excel2LaTeX from sheet 'Hoja1'
\begin{table}[H]
  \centering
    \begin{tabular}{|cc|r|}
    \rowcolor[rgb]{ .773,  .851,  .945} \multicolumn{3}{c}{\textbf{a 4 años}} \bigstrut[b]\\
    \hline
    \rowcolor[rgb]{ .773,  .851,  .945} \multicolumn{2}{|c|}{\textbf{VAN }} & \cellcolor[rgb]{ 1,  1,  1}\textcolor[rgb]{ 1,  0,  0}{-\$35.049,17} \bigstrut\\
    \hline
    \end{tabular}%
  \caption{VAN a 4 años.}  
  \label{tab:van-4}%
\end{table}%


La Tabla \ref{tab:van-4} indica un valor negativo del indicador VAN. Esto establece que el proyecto no es rentable en un lapso de 4 años, debido a que no hay un retorno positivo de caja. 

\subsubsection{VAN a 6 años}

En la Tabla \ref{tab:van-6} se muestra el resultado del cálculo de la VAN en un periodo de 6 años para el flujo de fondo operativo.

% Table generated by Excel2LaTeX from sheet 'Hoja1'
\begin{table}[H]
  \centering

    \begin{tabular}{|cc|r|}
    \rowcolor[rgb]{ .773,  .851,  .945} \multicolumn{3}{c}{\textbf{a 8 años}} \bigstrut[b]\\
    \hline
    \rowcolor[rgb]{ .773,  .851,  .945} \multicolumn{2}{|c|}{\textbf{VAN }} & \cellcolor[rgb]{ 1,  1,  1}\$7780,06 \bigstrut\\
    \hline
    \end{tabular}%
  \caption{VAN a 6 años.}    
  \label{tab:van-6}%
\end{table}%


El indicador arroja un valor positivo en 6 años. Esto se traduce en que el valor actual de los flujos es mayor al desembolso inicial en el octavo año.

En la primer etapa el proyecto no genera ningún tipo de retorno positivo, generando que la implementación del mismo sea inviable debido al exceso de tiempo que tarda en generar ganancias.

\subsection{TIR}

En cuanto a la TIR, es el indicador que se relaciona con el tipo de interés que provoca que la VAN sea cero. La función principal del mismo es la determinación de la tasa a la cual se recupera la inversión inicial del proyecto transcurrido cierto tiempo. Mientras mayor sea el valor de la TIR, más rentable es un proyecto. En caso de que su valor sea menor a lo esperado, el mismo indica que el proyecto no es rentable. Se realiza el análisis de rentabilidad del proyecto a partir del indicador TIR evaluado en un lapso de 4 y 6 años.

\subsubsection{TIR a 4 años}

En la Tabla \ref{tab:tir-4} se muestra el resultado del cálculo de la TIR en un periodo de 4 años para el flujo de fondos operativos indicado en la Tabla \ref{tab:flujo-fondos-operativos}.

% Table generated by Excel2LaTeX from sheet 'Hoja1'
\begin{table}[H]
  \centering
    \begin{tabular}{|cc|r|}
    \rowcolor[rgb]{ .773,  .851,  .945} \multicolumn{3}{c}{\textbf{a 4 años}} \bigstrut[b]\\
    \hline
    \rowcolor[rgb]{ .773,  .851,  .945} \multicolumn{2}{|c|}{\textbf{TIR}} & \cellcolor[rgb]{ 1,  1,  1}\textcolor[rgb]{ 1,  0,  0}{-12,441\%} \bigstrut\\
    \hline
    \end{tabular}%
  \caption{TIR a 4 años.}    
  \label{tab:tir-4}%
\end{table}%


EL indicador arroja un porcentaje mucho menor al esperado (1\%). Esto se traduce en que el proyecto es inviable en un periodo de 4 años.

\subsubsection{TIR en 6 años}

En la Tabla \ref{tab:tir-6} se muestra el resultado del cálculo de la TIR en un periodo de 8 años para el flujo de fondo operativos indicado en la Tabla \ref{tab:flujo-fondos-operativos}.

% Table generated by Excel2LaTeX from sheet 'Hoja1'
\begin{table}[H]
  \centering
    \begin{tabular}{|cc|r|}
    \rowcolor[rgb]{ .773,  .851,  .945} \multicolumn{3}{c}{\textbf{a 8 años}} \bigstrut[b]\\
    \hline
    \rowcolor[rgb]{ .773,  .851,  .945} \multicolumn{2}{|c|}{\textbf{TIR}} & \cellcolor[rgb]{ 1,  1,  1}3\% \bigstrut\\
    \hline
    \end{tabular}%
  \caption{TIR a 6 años.}    
  \label{tab:tir-6}%
\end{table}%


El indicador otorga un porcentaje igual al esperado. Esto indica que el proyecto es viable luego del transcurso de 6 años.