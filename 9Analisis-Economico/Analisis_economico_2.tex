Este capítulo presenta el análisis económico del proyecto, en base a los costos de inversión y los gastos de operación. Con los resultados obtenidos se determina la rentabilidad y sostenibilidad en el tiempo.

El análisis económico está dividido en la inversión en infraestructura CAPEX (Capital Expenditure), gastos de operación y mantenimiento OPEX (Operating Expenditure) y un plan de negocios. 

A continuación se mencionan las consideraciones adicionales para el desarrollo del proyecto: 

\begin{itemize}
\item En lo referente a la infraestructura de la red, los costos han sido obtenidos a través de diferentes proveedores Nacionales e Internacionales.
\item El análisis se realiza en dólares
\end{itemize}


\section{CAPEX}

El CAPEX (capital expenditure), es la inversión en capital que realiza la compañía para adquirir equipos, materiales, trabajos, etc. En resumen, es la inversión en la construcción de la red total para que esté lista para su funcionamiento. Este análisis es muy importante dentro de la actividad de la empresa y de su evolución futura.

Los costos del CAPEX que se considera es la compra del terreno para la construcción de la estación en Papagayos, considerando que la infraestructura en Carpintería ya se encuentra instalada. En lo referente a la red de acceso se considera el equipamiento activo y pasivo de la misma, incluyendo el costo de fibra óptica correspondientemente y el sistema de energía y protección. Finalmente se incluye el alquiler de infraestructura eléctrica.

Los costos para el proyecto se indican en la siguiente Tabla \ref{tab:capex}:

% Please add the following required packages to your document preamble:
% \usepackage[table,xcdraw]{xcolor}
% If you use beamer only pass "xcolor=table" option, i.e. \documentclass[xcolor=table]{beamer}
%\cellcolor[HTML]{C5D9F1}
\begin{table}[H]
\centering
\begin{tabular}{|l|l|}
\hline
\rowcolor[HTML]{C5D9F1}
Concepto                  & Costo          \\ \hline
%Red de Nucleo             & US\$ 155.248,48 \\ \hline
Red de Distribución       & US\$ 81.301,80   \\ \hline
Red de Acceso             & US\$ 13.685,30 \\ \hline
Mano de obra y Transporte & US\$ 16.980,08     \\ \hline
Costo Total del CAPEX     & US \$111.967,18 \\ \hline
\end{tabular}
  \caption{Costo del CAPEX.}
  \label{tab:capex}
\end{table}



%\FigureDEM{H}{10}{CAPEX}{Distribución CAPEX.}



\section{OPEX}

Son definidos como los gastos de operación y mantenimiento de la red, que se tendrán que realizar para tener el sistema en buen estado.

Los costos del OPEX se detallan a continuación:

\begin{itemize}
\item Costo de mantenimiento: Los costos se encuentran detallados en la seccion ANTERIOR y posee un valor de us\$ 1726,64

% \item Adquisición de datos y televisión a la empresa mayorista: se negocia un cotrato en el cual inicialmete se adquiere 500 Mbps a traves de un proveedor del mercado local(por ejemplo Telecom) y se compromete a la adquisición de  1 Gb en un plazo de 10 años. El costo mensual es de US\$ 1166, dando un total de US\$ 14000 anuales.
%----------

\item Servicio de datos mayorista:
Para la conexión del enlace troncal a internet se puede optar por dos alternativas
en general:
\begin{enumerate}
\item[•]Solicitar a un proveedor troncal el acceso.
\item[•]A partir de dos o más operadores de red, se comprometen acceso mutuo a sus
bases de clientes para el intercambio de tráfico, a esto se lo denomina ''peering".

Actualmente se efectúan acuerdos de peering y tránsito entre ISP's, proveedores
troncales regionales e internacionales y proveedores de contenido.
\end{enumerate}

% Los proveedores de red troncales del mercado local son los siguientes:
% \begin{enumerate}
% \item[•]ARLINK SA.
% \item[•]GLOBAL TECHNOLOGY S.R.L.
% \item[•]SIDECOM S.R.L.
% \item[•]TECOAR SA.
% \item[•]SILICA NETWORKS ARGENTINA SA.
% \item[•]NETWORKING SOLUTIONS
% \item[•]TELECOM.
% \end{enumerate}
%Fuentes: https://www.cabase.org.ar/nap-san-luis/
%Fuentes: http://listas.unsl.edu.ar/pipermail/unsl/2016-April/011142.html



Debido a que el peering es costoso, se tiende a agregar puntos
neutrales de interconexión, denominados ''plataformas de interconexión" (IXP).
Cabe destacar que también se los denomina NAP (Network Access Point).
%Fuente: http://papel.revistafibra.info/entrevista-con-ariel-graizer/#:~:text=Un%20NAP%20es%20un%20punto,es%20IXP%2C%20internet%20exchange%20point.&text=Un%20punto%20de%20intercambio%20de%20tr%C3%A1fico%20es%20aquel%20donde%20los,con%20otros%20operadores%20de%20redes.

Dicha plataforma facilita y reduce el costo del tráfico IP entre los distintos integrantes de este
ecosistema, a nivel regional, y a nivel internacional, y además mejora la calidad
del servicio.

En particular los proveedores mencionados anteriormente forman un IXP llamado 
''IXP San Luis", con lo cual, se decide gestionar la incorporación de la 
empresa a dicha plataforma.

El costo de adquisición del servicio se lo analiza de la siguiente forma:
El valor del tráfico de internet es distinto dependiendo de donde proviene o 
donde se lo va a buscar, por lo que el costo del trafico interno es diferente
al trafico externo.
Generalmente los proveedores no lo desglosan y cobran un valor uniforme.
Al formar parte de la IXP San Luis, se dividen 4 traficos diferentes mostrados
en la Tabla \ref{tab:trafico} junto con su porcentaje de uso aproximado.


% Table generated by Excel2LaTeX from sheet 'Hoja3'
\begin{table}[H]
  \centering
    \begin{tabular}{|c|c|c|}
    \hline
    \rowcolor[rgb]{ .773,  .851,  .945} \textbf{Tipo de trafico} & \textbf{Precio US\$ por Mbps} & \textbf{Desglose de tráfico} \bigstrut\\
    \hline
    Tráfico Local & US\$ 0 & 1\% \bigstrut\\
    \hline
    Tráfico entre IXP's & US\$ 27 & 25\% \bigstrut\\
    \hline
    Tráfico cachés & US\$ 0,2 a US\$5 & 31\% \bigstrut\\
    \hline
    Tráfico Internacional & US\$ 39 & 40\% \bigstrut\\
    \hline
    \end{tabular}%
	\caption{Tipos, Costos y Desglose de los tipos de tráfico.}
  \label{tab:trafico}%
\end{table}%

La estimación total del costo se observa en la Tabla \ref{tab:costo-internet} :



% \end{table}%

% Table generated by Excel2LaTeX from sheet 'Hoja3'
\begin{table}[H]
  \centering
    \begin{tabular}{|cc|c|}
\cline{1-2}    \rowcolor[rgb]{ .773,  .851,  .945} \multicolumn{1}{|c|}{\textbf{Tipo de trafico}} & \textbf{Magnitud Mbps} & \textbf{Costo US\$} \bigstrut\\
    \hline
    \multicolumn{1}{|c|}{Tráfico Local} & 0,53  & 0 \bigstrut\\
    \hline
    \multicolumn{1}{|c|}{Tráfico entre IXP's} & 13,25 & 357,750 \bigstrut\\
    \hline
    \multicolumn{1}{|c|}{Tráfico cachés} & 16,43 & 3,286 \bigstrut\\
    \hline
    \multicolumn{1}{|c|}{Tráfico Internacional} & 21,2  & 826,8 \bigstrut\\
    \hline
    \rowcolor[rgb]{ .773,  .851,  .945} \multicolumn{2}{|c|}{\textbf{Total}} & \textbf{1187,836} \bigstrut\\
    \hline
    \end{tabular}%
	\caption{Costo de servicio internet mayorista.}
  \label{tab:costo-internet}%
\end{table}%




\item Servicio de Televisión SDTV y HDTV:
El intercambio de datos de la TV digital se lo toma como tráfico proveniente de la caché almacenada.
La estimación total del costo se observa en la Tabla \ref{tab:costo-sdtv-hdtv} :




% Table generated by Excel2LaTeX from sheet 'Hoja3'
\begin{table}[H]
  \centering
    \begin{tabular}{|cc|c|}
\cline{1-2}    \rowcolor[rgb]{ .773,  .851,  .945} \multicolumn{1}{|c|}{\textbf{Tipo de trafico}} & \textbf{Magnitud Mbps} & \textbf{Costo US\$} \bigstrut\\
    \hline
    \multicolumn{1}{|c|}{Señal SDTV} & 30    & 6 \bigstrut\\
    \hline
    \multicolumn{1}{|c|}{Señal HDTV} & 60    & 12 \bigstrut\\
    \hline
    \rowcolor[rgb]{ .773,  .851,  .945} \multicolumn{2}{|c|}{\textbf{Total}} & \textbf{18} \bigstrut\\
    \hline
    \end{tabular}%
	\caption{Costo de servicio SDTV,HDTV mayorista.}
  \label{tab:costo-sdtv-hdtv}%
\end{table}%



\item Servicio de VoIP:
El intercambio de las señales de telefonía por medio de protocolo IP se lo toma como el promedio entre el tráfico entre IXP's e internacional.
La estimación total del costo se observa en la Tabla \ref{tab:costo-voip} :




% Table generated by Excel2LaTeX from sheet 'Hoja3'
\begin{table}[H]
  \centering
    \begin{tabular}{|cc|c|}
\cline{1-2}    \rowcolor[rgb]{ .773,  .851,  .945} \multicolumn{1}{|c|}{\textbf{Tipo de trafico}} & \textbf{Magnitud Mbps} & \textbf{Costo US\$} \bigstrut\\
    \hline
    \multicolumn{1}{|c|}{Servicio VoIP} & 1,83  & 40,26 \bigstrut\\
    \hline
    \rowcolor[rgb]{ .773,  .851,  .945} \multicolumn{2}{|c|}{\textbf{Total}} & \textbf{40,26} \bigstrut\\
    \hline
    \end{tabular}%
	\caption{Costo de servicio VoIP mayorista.}
  \label{tab:costo-voip}%
\end{table}%


%----------
\subsection{Proyecto de Negocio}
Se decide brindar tres tipos de planes para poder suscribirse y obtener el servicio que mejor se adapte a la necesidad del cliente. Los servicios que se ofrece a los clientes se ilustran en la Tabla \ref{tab:niv-serv}.
El costo promedio del servicio a un cliente en general, se lo ilustra en la Tabla \ref{tab:costo-servicio}.


% Table generated by Excel2LaTeX from sheet 'Hoja3'
\begin{table}[H]
  \centering
    \begin{tabular}{|c|c|}
    \hline
    \rowcolor[rgb]{ .773,  .851,  .945} \multicolumn{2}{|c|}{\textbf{Costo Promedio del Servicio en US\$}} \bigstrut\\
    \hline
    Costo/cliente prom. internet & 19,473 \bigstrut\\
    \hline
    Costo/cliente prom. SDTV-HDTV & 0,295 \bigstrut\\
    \hline
    Costo/cliente prom VoIP & 0,660 \bigstrut\\
    \hline
    \rowcolor[rgb]{ .773,  .851,  .945} \textbf{Costo Promedio} & \textbf{20,43} \bigstrut\\
    \hline
    \end{tabular}%
	\caption{Costo del servicio.}
  \label{tab:costo-servicio}%
\end{table}%


Es posible advertir en la Tabla anterior que la composición del costo del servicio
se basa principalmente en la provisión del servicio de internet.

Cabe destacar que para cubrir el costo del servicio por cliente, debe cobrarse al menos US\$ 40,86.
Dicho precio se encuentra dentro del rango que una familia de la localidad asigna a las TICs.

\subsubsection{Planes para suscripción}
El costo de los servicios que se ofrece a los clientes se ilustran en la Tabla \ref{tab:precio-planes}.


% Table generated by Excel2LaTeX from sheet 'Hoja3'
\begin{table}[H]
  \centering
    \begin{tabular}{|c|c|}
    \hline
    \rowcolor[rgb]{ .773,  .851,  .945} \textbf{Plan} & \textbf{Precio US\$} \bigstrut\\
    \hline
    Bronce & 32,69 \bigstrut\\
    \hline
    Plata & 40,86 \bigstrut\\
    \hline
    Oro   & 46,99 \bigstrut\\
    \hline
    \end{tabular}%
    \caption{Precio de los planes.}
  \label{tab:precio-planes}%
\end{table}%


El costo de los servicios se coloca en base al costo promedio de gastos que se detalla en la Tabla \ref{tab:costo-servicio}. Se toma como precio del plan plata al costo promedio de la Tabla \ref{tab:costo-servicio}, al plan bronce un 20\% más barato que el plan plata y al plan oro el cual tiene todos los beneficios es un 15\% mas caro que el plan que se toma de referencia. 

Los costo de los servicios poseen precios semejantes a las  empresas competidoras que brindan el servicio
Triple Play.


\subsubsection{Ingresos Extras}
Otra fuente importante de ingresos corresponde a las pautas publicitarias, la Tabla \ref{tab:tarifas-publicidad} ilustra las tarifas
que se cobra y el ingreso mensual estimado, considerando que la duración de las mismas es de 25 segundos
y se repetirán 10 veces al día (excepto las publicidades políticas, las cuales se consideran 20 veces por día).

% Table generated by Excel2LaTeX from sheet 'Hoja4'
\begin{table}[H]
  \centering
  \tiny
    \begin{tabular}{|cccc|c|}
    \hline
    \rowcolor[rgb]{ .773,  .851,  .945} \multicolumn{1}{|c|}{\textbf{Publicidad}} & \multicolumn{1}{c|}{\textbf{US\$ x seg}} & \multicolumn{1}{c|}{\textbf{US\$ x 25 segundos}} & \textbf{Cantidad mensual} & \textbf{Tarifa US\$} \bigstrut\\
    \hline
    \multicolumn{1}{|l|}{\textbf{Rotativo Comercial}} & \multicolumn{1}{c|}{0,03} & \multicolumn{1}{c|}{0,85} & 310   & 263,25 \bigstrut\\
    \hline
    \multicolumn{1}{|l|}{\textbf{Rotativo (25 seg)}} & \multicolumn{1}{c|}{0,64} & \multicolumn{1}{c|}{15,89} & 310   & 4925,88 \bigstrut\\
    \hline
    \multicolumn{1}{|l|}{\textbf{Digitado Comercial Mañana}} & \multicolumn{1}{c|}{0,06} & \multicolumn{1}{c|}{1,38} & 310   & 427,51 \bigstrut\\
    \hline
    \multicolumn{1}{|l|}{\textbf{Digitado Comercial Tarde y Noche}} & \multicolumn{1}{c|}{0,09} & \multicolumn{1}{c|}{2,33} & 310   & 722,35 \bigstrut\\
    \hline
    \multicolumn{1}{|l|}{\textbf{Digitado TOP}} & \multicolumn{1}{c|}{0,08} & \multicolumn{1}{c|}{2,12} & 310   & 657,07 \bigstrut\\
    \hline
    \multicolumn{1}{|l|}{\textbf{Publicidad Oficial,Politicas y Nac}} & \multicolumn{1}{c|}{0,25} & \multicolumn{1}{c|}{1,27} & 620   & 788,48 \bigstrut\\
    \hline
    \rowcolor[rgb]{ .773,  .851,  .945} \multicolumn{4}{|c|}{\textbf{Total mensual}} & \textbf{7784,54} \bigstrut\\
    \hline
    \end{tabular}%
	\caption{Tarifas de publicidad.}
  \label{tab:tarifas-publicidad}%
\end{table}%


\end{itemize}


Otra fuente de ingreso que se considera agregar corresponde al alquiler de fibra optica oscura hacia futuros clientes tales como ''Autopista de la Información$"$
y una empresa avícola radicada en las cercanías de Papagayos. Para determinar el precio del alquiler se recurre a referencias tomadas a partír de investigaciones.
%https://www.mitic.gov.py/application/files/5415/5646/4485/Internacional_Argentina_1.pdf

El ingreso anual se determina en la Ecuación , teniendo en cuenta que el precio unitario por kilometro (PU) es igual a $\dfrac{US\$ 400}{km}$.
\begin{equation}
 Costo_{anual} = PU \cdot Distancia \\ 
 = US\$ 12000
\end{equation}



%\end{itemize}

%http://desarrollo.sanluisctv.com.ar/index.php/publicidad


%https://www.itu.int/en/ITU-D/Regulatory-Market/Documents/CostaRica/Presentations/Session5_Messano_IXP.pdf


%https://www.cessi.org.ar/descarga-institucionales-2330/documento2-290a5d80ccf2da8c5d3935209aa10568

\begin{thebibliography}{99}
\bibitem{alquiler-fo-oscura} Alquiler FO oscura: \begin{tiny}
\url{https://www.mitic.gov.py/application/files/5415/5646/4485/Internacional_Argentina_1.pdf}
\end{tiny}

\bibitem{publicidad} Tarifas Publicidad: \begin{tiny}
\url{http://desarrollo.sanluisctv.com.ar/index.php/publicidad}
\end{tiny}

\bibitem{publicidad} Tarifas Trafico Internet: \begin{tiny}
\url{https://www.itu.int/en/ITU-D/Regulatory-Market/Documents/CostaRica/Presentations/Session5_Messano_IXP.pdf}
\end{tiny}

\bibitem{ixpsanluis} IXP San Luis:
\begin{tiny}
\url{https://www.cabase.org.ar/nap-san-luis/}
\end{tiny}

\bibitem{hubcontenidos} Hub Contenidos CABASE:
\begin{tiny}
\url{https://www.cabase.org.ar/wp-content/uploads/2020/01/Folleto-Hub-de-Contenidos_c-1.pdf}
\end{tiny}

\bibitem{informe} Reporte Anual del Sector de Software y Servicios Informaticos de la República Argentina (2018):


\begin{tiny}
\url{https://www.cessi.org.ar/descarga-institucionales-2330/documento2-290a5d80ccf2da8c5d3935209aa10568}
\end{tiny}

\end{thebibliography}


 \begin{figure}[H]
	 \centering
	 \includegraphics[angle=90, scale=0.8]{dem_econom/figuras/dem/mod_flujos_2}
	 \caption{Modelo de Flujos de Fondos a Terceros.}
	 \label{dem-figuras:mod_flujos}

 \end{figure}


