\section{Enlace troncal por Fibra Óptica}

Como ya se mencionó, el despliegue del mismo se realizará en forma aérea siguiendo la  trayectoria de la Figura \ref{dem-figuras:fig2004} que une la oficina central ubicada en la localidad de Carpintería con la localidad de Papagayos, sobre un tendido eléctrico pre existente perteneciente a la empresa EDESAL, el cual se extiende en las cercanías  del recorrido de las rutas.

\subsection{Tendido eléctrico}

Los postes del tendido eléctrico en las cercanías de la Ruta Nacional N{$^{\circ}$}1) pertenecientes a la empresa, poseen una distancia aproximada de 64 m entre postes adyacentes. Esto significa que para una extensión de 30,2 Km (distancia correspondiente entre la localidad de Papagayos y Carpintería a través de la Ruta Provincial N{$^{\circ}$}1), hay, aproximadamente, un total de 472 postes.


Para acceder a la utilización de los postes, se procede al alquiler de los mismos. El precio de cada poste es de \$37 por mes más impuestos. Aproximando, se calcula un total de \$50 por poste por mes. Se opta por el alquiler de los postes, ya que la instalación de los mismos por cuenta propia implica un costo mayor.

\subsection{Tipo de Fibra Óptica}

La longitud requerida del cable óptico que va montado sobre la red eléctrica, es de 33,22 Km, el cual incluye el tendido sobre los postes, la catenaria y el margen de seguridad para futuras reparaciones. La fibra óptica debe ser autosoportado ADSS ya que la instalación de la misma es del tipo aérea entre los postes eléctricos.

La fibra que se utiliza es Belden M9W510T, la cual posee 6 pelos y un costo de US\$1,7 por metro.

%Los rollos de la fibra Belden M9W510T son de 4 Km, por lo que para cubrir los 33,22 Km del tendido son necesarios 9 rollos, teniéndose un sobrante aproximado de 2,8 Km apuntado a futuras reparaciones. Cada rollo tiene un costo de US\$6800 con el envío incluido \cite{adss}.

Para realizar las uniones correspondientes entre los pelos de fibra de  cada rollo de cable óptico, se necesitan 8 cajas de empalmes. Además se requieren dos cajas  extras para efectuar la bajada de fibra del tendido eléctrico y proceder con la terminación de la misma en los racks ubicados en Carpintería y Papagayos. De esta forma, son requeridas 10 cajas de empalme.
El modelo elegido es la caja de empalme FCBD12 con un precio de US\$ 72,38, el cual se muestra en la Figura \ref{dem-figuras:fig1001}. Las características de la caja se encuentran en el Apéndice B.
\FigureDEM{H}{5}{fig1001}{Caja de empalmes}


\subsection{Instalaciónes Físicas}


Para la protección de los equipos y dispositivos involucrados en el despliegue de la red óptica, estos se colocan e instalan en el interior de una caseta. La misma posee una dimensión de 4m2 con una ventana de 1m2. La construcción de la misma es con material para asegurar que el deterioro de la misma sea lo menor posible frente a daños intencionados o provocados por condiciones climáticas. El valor de la misma es de US\$528,19 \cite{modplant3}. La Figura \ref{dem-figuras:fig1006} muestra un modelo de caseta.


\FigureDEM{H}{6}{fig1006}{Caseta}


Un terreno de 36m2 ubicado en la localidad de Papagayos para la construcción de la caseta tiene un precio aproximado de US\$ 891 \cite{terreno1duplicado} \cite{terreno2duplicado} \cite{terreno3duplicado}.

\subsection{Resumen de Costos}

En la Tabla \ref{tab:Costos_int_FO} se listan los diversos elementos comprados a proveedores internacionales, a cuyos precios se les incrementa un 40\% a causa de impuestos, costos de importación, etc.

% Table generated by Excel2LaTeX from sheet 'Hoja1'
\begin{table}[htbp]
  \centering
    \begin{tabular}{|r|r|l|r|}
    \hline
    \rowcolor[HTML]{C5D9F1} \multicolumn{1}{|l|}{\textbf{Descripción}} & \multicolumn{1}{l|}{\textbf{Cantidad}} & \textbf{Origen de compra} & \multicolumn{1}{l|}{\textbf{Precio Total}} \bigstrut\\
    \hline
    \multicolumn{1}{|l|}{Fibra Óptica} & \multicolumn{1}{c|}{33220 m} & Internacional & \multicolumn{1}{l|}{US\$ 56.485} \bigstrut\\
    \hline
    \multicolumn{1}{|l|}{Conector SC/APC} & \multicolumn{1}{c|}{2} & Internacional & \multicolumn{1}{l|}{US\$ 1,48} \bigstrut\\
    \hline
    \multicolumn{1}{|l|}{Caja empalme} & \multicolumn{1}{c|}{10} & Internacional & \multicolumn{1}{l|}{US\$ 723,8} \bigstrut\\
    \hline
          &       & \textbf{Costo Total} & \multicolumn{1}{l|}{\textbf{US\$ 57.210,28}} \bigstrut\\
    \hline
    \rowcolor[HTML]{C5D9F1} \multicolumn{1}{|l|}{\textbf{Costos de Importación}} &       &       &  \bigstrut\\
    \hline
    \multicolumn{1}{|l|}{IVA Tasa General} & 10\%  &  -    & \multicolumn{1}{l|}{US\$ 5.721,02} \bigstrut\\
    \hline
    \multicolumn{1}{|l|}{IVA Adicional} & 20\%  & Resolución AFIP 3373/2012 & \multicolumn{1}{l|}{US\$ 11442,06} \bigstrut\\
    \hline
    \multicolumn{1}{|l|}{Impuesto a las Ganancias} & 6\%   & Resolución AFIP 3373/2012 & \multicolumn{1}{l|}{US\$ 3.432,61} \bigstrut\\
    \hline
    \multicolumn{1}{|l|}{Ingresos Brutos} & 3\%   & Resolución AFIP 3373/2012 & \multicolumn{1}{l|}{US\$ 1.716,30} \bigstrut\\
    \hline
    \multicolumn{1}{|l|}{Tasa de Oficialización} & \multicolumn{1}{l|}{US\$ 10,00} & Aduana & \multicolumn{1}{l|}{US\$ 10} \bigstrut\\
    \hline
    \multicolumn{1}{|l|}{Tasa de Digitalización} & \multicolumn{1}{l|}{US\$ 28,00} & Aduana & \multicolumn{1}{l|}{US\$ 28} \bigstrut\\
    \hline
    \rowcolor[HTML]{C5D9F1}       &       & \textbf{Costo Final} & \textbf{US\$ 79.560,27} \bigstrut\\
    \hline
    \end{tabular}%
    \caption{Costos productos Internacionales. Enlace Troncal Fibra Óptica}
  \label{tab:Costos_int_FO}%
\end{table}%




La Tabla \ref{tab:Costos_nac_FO} indican los elementos comprados a proveedores nacionales, cuyo precio se expresa en US\$. 



% Table generated by Excel2LaTeX from sheet 'Hoja1'
\begin{table}[htbp]
  \centering
    \begin{tabular}{|r|r|l|l|}
    \hline
    \rowcolor[HTML]{C5D9F1} \multicolumn{1}{|l|}{\textbf{Descripción}} & \multicolumn{1}{l|}{\textbf{Cantidad}} & \textbf{Origen de Compra} & \textbf{Precio Total} \bigstrut\\
    \hline
    \multicolumn{1}{|l|}{Alquiler Postes} & 472   & Nacional & US\$ 236 (mensual) \bigstrut\\
    \hline
    \multicolumn{1}{|l|}{Caseta} & 1     & Nacional & US\$ 528,19 \bigstrut\\
    \hline
    \multicolumn{1}{|l|}{Terreno} & 1     & Nacional & US\$ 875,16 \bigstrut\\
    \hline
    \rowcolor[HTML]{C5D9F1}       &       & \textbf{Costo Total} & \textbf{US\$ 1639,35} \bigstrut\\
    \hline
    \end{tabular}%
\caption{Costos productos Nacionales. Enlace Troncal Fibra Optica.}
  \label{tab:Costos_nac_FO}%
\end{table}%



Para el despliegue de la red óptica se requiere un total de  US\$ 81.199,62 , valor que se obtiene sumando los totales de las Tablas \ref{tab:Costos_int_FO} y \ref{tab:Costos_nac_FO}.


\newpage
