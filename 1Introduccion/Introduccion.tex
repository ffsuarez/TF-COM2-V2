El presente trabajo consiste en la implementación de una red de telecomunicaciones que permite otorgar un servicio de Triple Play a la ciudad de Papagayos desde la ciudad de Carpinteria, ubicadas sobre la Provincia de San Luis. De manera que la empresa ficticia se instalada sobre la ciudad de Carpintería.

El desarrollo del trabajo se ha dividido en 9 partes, sobre la parte \ref{demografico} se exhibe un análisis demográfico de la localidad sobre la que se plantea brindar un servicio de comunicación del tipo " Triple Play''$,$ en la cual se visualiza la situación socio-económica de la ciudad, y permitiendo definir de la forma más exacta posible el monto que una familia destina para las TIC's.

En la parte \ref{tecnologico} se evalúan las diversas tecnologías para establecer y facilitar la comunicación entre las dos localidades. Se analizan las posibles formas de establecer las Redes de  Distribución y Acceso. Cabe destacar que el análisis técnico exhaustivo se presenta de forma adjunta a este informe.

De acuerdo a las tecnologías planteadas, en la parte \ref{plan-economico} se efectúa un analisis económico de las mismas, para  determinar los diversos equipos que pueden utilizarse de acuerdo a las opciones planteadas en la parte anterior y además se exhiben los costos en los que se incurre para la adquisición de los equipos necesarios.

En el apartado \ref{propuesta-tecno} se elige una de las alternativas propuestas para establecer dicha red, considerando los factores más críticos que intervienen. Mientras que la parte \ref{plan-accion} desarrolla la planificación necesaria para la puesta en funcionamiento inicial de la red, se describe la metodología, el personal que se precisa y los costos necesarios.

Sobre la parte \ref{plan-mantenimiento} se expresa un plan de mantenimiento preventivo en la que se permite observar como se programa la mantención y supervisión de la red, con el propósito de reducir lo más posible la cantidad de fallas o degradaciones de la calidad de funcionamiento.

Una vez determinado la elección de equipos, costos y plan de mantenimiento se desarrolla en la parte \ref{analisis-economico} el análisis económico del negocio en el que se describe la inversión necesaria para la puesta en funcionamiento y la existencia de la rentabilidad de la solución elegida.

Para finalizar se expresan en la parte \ref{conclusion} los conceptos que se aplicaron, la forma en que se relacionan, y se determina en que medida es posible aplicar esta inversión a partír de los análisis económicos y técnicos que se desarrollan a lo largo del informe.