\section{Anexo A: Análisis Salario Bruto Promedio y Porcentaje de Inversión en TIC's}

%\bibitem{base-datos}Base de datos sobre sueldos:{\tiny  \url{https://www.lovemondays.com.ar/}}
%
%\bibitem{jub-media}ANSES Datos Abiertos. Junio 2019: {\tiny \url{https://www.anses.gob.ar/institucional/datos-abiertos}} 

% Archivo en formato xlsx. Celda B83.
%
%\bibitem{inflacion}Cálculo de inflación realizado: {\tiny \url{https://calculadoradeinflacion.com/argentina.html?md=agosto&ad=2018&mh=septiembre&ah=2019&q=31400}}
%
%\bibitem{drive}Apendice A compartido en drive: {\tiny \url{https://drive.google.com/drive/folders/1vu527bN0kaPhcQiW2mn9SPwRCseGwazx?usp=sharing}}



Para realizar una estimación del monto destinado a las diversas tecnologías de las telecomunicaciones que puede realizar un hogar en la localidad, se calcula un valor promedio del ingreso por habitante de la ciudad, el cual se encuentra formado a su vez del salario  bruto promedio en base a información obtenida a través de la base de datos consultada \cite{base-datos} y también por la jubilación o pensión mínima percibida. Cabe destacar que la gran mayoría de los salarios consultados corresponden al período de noviembre de 2018 \cite{drive}.

En base al valor promedio de ingreso por habitante de ciudad, se plantean 4 modelos de hogares que se caracterizan por el numero de habitantes, de manera que el modelo 1 se asocia a 1 persona, el modelo 2 se asocia a 2 habitantes y así sucesivamente.

Para cada modelo se observa distintos casos posibles a partir de la información de la conformación del hogar, obtenido a través de la base de datos del Censo 2010 y se calcula un ingreso promedio, de manera que este ultimo valor se asocia con el ingreso bruto por hogar constituido por n habitantes (n=1,2,3,4,5).

Para el valor promedio por habitante se tiene presente 
que el ingreso medio por trabajador calculado a partir de la Tabla \ref{tab:sueldos} es de \$22.908,21.
Mientras que para promediar el valor medio de una jubilación en las cercanías de la localidad se toma como referencia el haber medio de Jubilaciones y Pensiones por tipo de prestación, en el cual afirma que la jubilación media provincial en junio de 2019 es de \$ 18.075 \cite{jub-media}, cuyo valor actual (noviembre 2020) equivale aproximadamente a \$ 29.586 \cite{inflacion}, sin embargo se estima \$ 27.500 para brindar consistencia con el ingreso medio de un trabajador. Por lo tanto, el monto del ingreso por habitante calculado es de \$ 25.204,10 . Se considera que el monto de la asignación universal por hijo es \$ 2.652.



%Table generated by Excel2LaTeX from sheet 'Hoja1'
\begin{table}[H]
  \centering\scriptsize
    \begin{tabular}{|c|c|c|}
    \hline
    \textcolor[rgb]{ .122,  .286,  .49}{\textbf{Institucion/Empresa}} &
      \textcolor[rgb]{ .122,  .286,  .49}{\textbf{Cargo}} &
      \textcolor[rgb]{ .122,  .286,  .49}{\textbf{Salario Bruto}}
      \bigstrut\\
    \hline
    Consejo de Federaciones de Bomberos Voluntarios &
      Obrero Contratado &
      \$ 18.500,00
      \bigstrut\\
    \hline
    YPF &
      Playero &
      \$ 18.805,00
      \bigstrut\\
    \hline
    Black \& Decker &
      Operario &
      \$ 13.680,00
      \bigstrut\\
    \hline
    Ecogas &
      Asistente Administrativo &
      \$ 26.500,00
      \bigstrut\\
    \hline
    Rovella Carranza &
      Auxiliar Administrativo &
      \$ 18.349,00
      \bigstrut\\
    \hline
    Escuela Primaria &
      Docente &
      \$ 25.100,00
      \bigstrut\\
    \hline
    Ministerio de Hacienda y Finanzas Publicas &
      Administrativo &
      \$ 18.317,00
      \bigstrut\\
    \hline
    Ministerio de Salud de la Nacion &
      Administrativo &
      \$ 16.090,00
      \bigstrut\\
    \hline
    Ministerio de Trabajo de la Nacion &
      Asistente Administrativo &
      \$ 17.475,00
      \bigstrut\\
    \hline
    Ministerio de Transporte &
      Asistente Tecnico &
      \$ 20.982,00
      \bigstrut\\
    \hline
    Obra Social de Empleados Públicos &
      Empleado Administrativo &
      \$ 21.000,00
      \bigstrut\\
    \hline
    Poder Judicial de la Provincia de San Luis &
      Auxiliar  &
      \$ 16.850,00
      \bigstrut\\
    \hline
    Policia de San Luis &
      Auxiliar &
      \$ 29.600,00
      \bigstrut\\
    \hline
    RENATRE(Registro Nacional de Trabajadores Rurales y Empleadores) &
      Maquinista Especializado &
      \$ 25.100,00
      \bigstrut\\
    \hline
    RENATRE(Registro Nacional de Trabajadores Rurales y Empleadores) &
      Empleado Rural &
      \$ 18.100,00
      \bigstrut\\
    \hline
    Agricultores Federados Argentinos &
      Ingeniero Agrónomo &
      \$ 30.500,00
      \bigstrut\\
    \hline
    UTHGRA &
      Escala Salarial n$^{\circ}$1 &
      \$ 21.957,00
      \bigstrut\\
    \hline
    UTHGRA &
      Escala Salarial n$^{\circ}$2 &
      \$ 23.042,00
      \bigstrut\\
    \hline
    UTHGRA &
      Escala Salarial n$^{\circ}$3 &
      \$ 24.176,00
      \bigstrut\\
    \hline
    UTHGRA &
      Escala Salarial n$^{\circ}$4 &
      \$ 25.633,00
      \bigstrut\\
    \hline
    UTHGRA &
      Escala Salarial n$^{\circ}$5 &
      \$ 27.182,00
      \bigstrut\\
    \hline
    UTHGRA &
      Escala Salarial n$^{\circ}$6 &
      \$ 32.596,00
      \bigstrut\\
    \hline
    UTHGRA &
      Escala Salarial n$^{\circ}$7 &
      \$ 38.075,00
      \bigstrut\\
    \hline
    \end{tabular}%
    \caption{Sueldos obtenidos en base de datos.}
  \label{tab:sueldos}%
\end{table}%


















En función de los valores previamente calculados, se realiza el análisis del ingreso monetario por hogar de la localidad estudiada:

\begin{itemize}
\item Modelo 1: Un habitante por hogar

Este modelo corresponde a un ingreso por hogar de \$ 25204.10

\item Modelo 2: Dos habitantes por hogar

El ingreso por hogar, considerando que una sola persona es económicamente activa es \$ 25.204,10 .

Mientras que si el hogar se encuentra formado por 2 personas económicamente activas el monto total es \$50.408,21. 

El promedio de ingreso monetario es de \$ 37.806,16.

\item Modelo 3: Tres habitantes por hogar

Se consideran los siguientes casos:
\begin{enumerate}
\item[•]Una pareja con un hijo menor:

En caso que ambas personas trabajen, al incluir el monto de la asignación universal el resultado del aporte total es \$ 53.060,21. 
% $ \$50408.2 + 2652 = 

Si solo 1 persona trabaja, el monto desciende a  \$ 27.856,1 .

El promedio obtenido es  \$ 40.458,16.

\item[•]Una pareja con un hijo mayor:

Considerando que ambas personas trabajan el monto es  \$ 50.408,2.

Si se considera que 1 sola persona trabaja, el valor corresponde al modelo 1.

El monto promedio en este caso particular es  \$ 32.763,55 .
\end{enumerate}
El monto promedio de un hogar habitado por 3 personas es \$ 36.610,85 .
Se considera que el hogar se encuentra constituido por la pareja y un hijo/a, se analiza el caso que ambas personas trabajan y , por lo tanto en el  monto total se incluye la asignación universal por hijo. El resultado del aporte total es de $ \$50408.2 + 2652 = \$ 53060.21$. 

Mientras que al analizar el caso en el que solo 1 habitante es economicamente activo, 1 inactivo, y uno de ellos cobra la asignación universal, corresponde a un ingreso total es \$27856.10.

El promedio total indica que el ingreso por un hogar constituido por 2 habitantes es de \$40458.16.
 
\item Modelo 4: Cuatro habitantes por hogar

Los casos que se plantean son los siguientes:
\begin{itemize}
\item Una pareja con 2 hijos menores:

Suponiendo que solo 1 persona trabaja , 1 es desocupada y se cobra pensión por ambos menores, el ingreso corresponde a \$ 30.508,10, mientras que si ambos trabajan y se cobra pensión por los dos hijos el monto asciende a \$ 55.712,21 . 

El promedio indica que el ingreso por hogar en este caso es \$43.110,16.

\item Una pareja con 1 hijo mayor de edad y 1 menor de edad:

Considerando que 3 personas trabajan y se cobra la asignación por el menor, el ingreso es \$ 78.264,32 .

Considerando el caso que 2 personas se encuentran ocupadas, 1 de ellas desocupada y se cobra la pensión por el menor, el monto disminuye a \$ 53.060,21. 

Finalmente si solo 1 persona trabaja y las 2 restantes se encuentran desocupadas, y se cobra la asignación universal por hijo,el monto es \$27.586,10. 

El ingreso promedio en este caso particular es \$53.060,21.

\item Una pareja con 1 persona jubilada y 1 hijo menor:

Suponiendo que solo 1 persona trabaja, al considerar el cobro de la jubilación y la asignación por el menor, el ingreso es \$ 53.060,21 .

Si 2 personas trabajan, al considerar el beneficio de la jubilación y el monto cobrado por el menor la suma asciende a \$ 78.264,32. 

El ingreso promedio en este caso particular es de \$65.662,27.

\item Una pareja con 1 persona jubilada y 1 hijo mayor que no trabaja:

Suponiendo que solo 1 persona trabaja, 1 de ellas se encuentra desocupada, al considerar el monto cobrado por la jubilación, el ingreso es \$ 50.408,21.

Si 2 personas trabajan, teniendo en cuenta el cobro de la jubilación el monto asciende a \$ 75.612,32. 

El ingreso promedio en este caso es \$63.010,27.
\end{itemize}

El aporte medio de un hogar de 4 personas es  \$ 56.210,72 .


\end{itemize}


%
La Tabla \ref{tab:salarios} resume los aportes medios calculados anteriormente.

% Table generated by Excel2LaTeX from sheet 'Hoja2'
\begin{table}[H]
  \centering
  \tiny  
    \begin{tabular}{|c|c|}
    \hline
    \textcolor[rgb]{ .122,  .286,  .49}{\textbf{Cantidad de habitantes}} &
      \textcolor[rgb]{ .122,  .286,  .49}{\textbf{Aporte Medio}}
      \bigstrut\\
    \hline
    \textbf{1} &
       \$ 25.204,1 
      \bigstrut\\
    \hline
    \textbf{2} &
      \$ 37.806,16
      \bigstrut\\
    \hline
    \textbf{3} &
      \$ 36.610,55
      \bigstrut\\
    \hline
    \textbf{4} &
      \$ 56.210,72
      \bigstrut\\
    \hline
    \end{tabular}%
    \caption{Ingresos Medios por cantidad de habitantes por hogar.}
  \label{tab:salarios}%
\end{table}%

%
%
A partír de dichos salarios promedios calculados se pretende determinar cuál de los 5 modelos corresponde al más representativo del mercado al que se pretende prestar el servicio, en este caso particular se decide analizar el Departamento Chacabuco.

El análisis considera el porcentaje de uso de teléfono celular y computadora, en función del número de habitantes por hogar. Dichos valores se pueden observar en las Tablas \ref{tab:celular} y \ref{tab:computadora}.



% Table generated by Excel2LaTeX from sheet 'Hoja1'
\begin{table}[H]
  \centering
  \tiny

    \begin{tabular}{|c|c|c|c|}
    \hline
    \multirow{2}[4]{*}{\textcolor[rgb]{ .122,  .286,  .49}{\textbf{Total de personas en el hogar}}} &
      \multicolumn{2}{c|}{\textcolor[rgb]{ .122,  .286,  .49}{\textbf{Telefono Celular}}} &
      \multicolumn{1}{c|}{\multirow{2}[4]{*}{\textcolor[rgb]{ .122,  .286,  .49}{\textbf{Total}}}}
      \bigstrut\\
\cline{2-3}     &
      \textcolor[rgb]{ .122,  .286,  .49}{\textbf{Si}} &
      \textcolor[rgb]{ .122,  .286,  .49}{\textbf{No}} &
      
      \bigstrut\\
    \hline
    \textbf{1} &
      11,25 \% &
      6,73 \% &
      17,98 \%
      \bigstrut\\
    \hline
    \textbf{2} &
      17,17 \% &
      4,86 \% &
      22,04 \%
      \bigstrut\\
    \hline
    \textbf{3} &
      17,39 \% &
      1,98 \% &
      19,37 \%
      \bigstrut\\
    \hline
    \textbf{4} &
      17,95 \% &
      1,51 \% &
      19,46 \%
      \bigstrut\\
    \hline
    \textbf{5} &
      10,29 \% &
      0,92 \% &
      11,2 \%
      \bigstrut\\
    \hline
    \textbf{6} &
      5,19 \% &
      0,62 \% &
      5,81 \%
      \bigstrut\\
    \hline
    \textbf{7} &
      2,03 \% &
      0,22 \% &
      2,24 \%
      \bigstrut\\
    \hline
    \textbf{Más} &
      1,7 \% &
      0,19 \% &
      1,89 \%
      \bigstrut\\
    \hline
    \end{tabular}%
  \caption{Uso de celulares por cantidad de habitantes en el hogar. Departamento Chacabuco.}
  \label{tab:celular}%
\end{table}%

%
%
%
%
%
%
%
%
%
%
%
%
%
%
%
%
%
%
%
% Table generated by Excel2LaTeX from sheet 'Hoja1'
\begin{table}[H]
  \centering
  \tiny
    \begin{tabular}{|c|c|c|c|}
    \hline
    \multirow{2}[4]{*}{\textcolor[rgb]{ .122,  .286,  .49}{\textbf{Total de personas en el hogar}}} &
      \multicolumn{2}{c|}{\textcolor[rgb]{ .122,  .286,  .49}{\textbf{Computadora}}} &
      \multicolumn{1}{c|}{\multirow{2}[4]{*}{\textcolor[rgb]{ .122,  .286,  .49}{\textbf{Total}}}}
      \bigstrut\\
\cline{2-3}     &
      \textcolor[rgb]{ .122,  .286,  .49}{\textbf{Si}} &
      \textcolor[rgb]{ .122,  .286,  .49}{\textbf{No}} &
      
      \bigstrut\\
    \hline
    \textbf{1} &
      3,1 \% &
      14,88 \% &
      17,98 \%
      \bigstrut\\
    \hline
    \textbf{2} &
      7,22 \% &
      14,82 \% &
      22,04 \%
      \bigstrut\\
    \hline
    \textbf{3} &
      10,57 \% &
      8,8 \% &
      19,37 \%
      \bigstrut\\
    \hline
    \textbf{4} &
      12,65 \% &
      6,81 \% &
      19,46 \%
      \bigstrut\\
    \hline
    \textbf{5} &
      7,53 \% &
      3,68 \% &
      11,2 \%
      \bigstrut\\
    \hline
    \textbf{6} &
      3,65 \% &
      2,17 \% &
      5,81 \%
      \bigstrut\\
    \hline
    \textbf{7} &
      1,32 \% &
      0,92 \% &
      2,24 \%
      \bigstrut\\
    \hline
    \textbf{Más} &
      0,94 \% &
      0,95 \% &
      1,89 \%
      \bigstrut\\
    \hline
    \end{tabular}%
  \caption{Uso de computadora por cantidad de habitantes en el hogar. Departamento Chacabuco.}
  \label{tab:computadora}%
\end{table}%

%
%
%
%
%
Es posible afirmar que un grupo importante del mercado que presenta una mayor tendencia de uso de las TIC's corresponde a los hogares compuestos por 3 o 4 habitantes, a raíz de esto, se define que el porcentaje del monto estimado de uso de las TIC's se calcula utilizando el ingreso medio de un hogar compuesto por 3 personas.

%
Los precios de referencia se toman a partir de los valores consultados en la pagina web de la ENACOM (Ente Nacional de las Comunicaciones).
%poner cita bibliografica
Los mismos se exhiben en la Tabla \ref{tab:planes}.










% Table generated by Excel2LaTeX from sheet 'Hoja2'
\begin{table}[H]
  \centering
  \tiny  
    \begin{tabular}{|c|c|c|r}
    \hline
    \textcolor[rgb]{ .122,  .286,  .49}{\textbf{Nombre del plan}} &
      \textcolor[rgb]{ .122,  .286,  .49}{\textbf{Empresa}} &
      \textcolor[rgb]{ .122,  .286,  .49}{\textbf{Abono Nominal Conjunto}} &
      \multicolumn{1}{c|}{\textcolor[rgb]{ .122,  .286,  .49}{\textbf{Localidad}}}
      \bigstrut\\
    \hline
    \textbf{Cable + 10 MB} &
      CablenetTV &
      \textbf{\$1.640} &
      \multicolumn{1}{c|}{Santa Clara-Prov. De Santa Fe}
      \bigstrut\\
    \hline
    \textbf{Cable + 10 MB} &
      CablenetTV &
      \textbf{\$1.195} &
      \multicolumn{1}{c|}{San Martin-Prov. De Santa Fe}
      \bigstrut\\
    \hline
    10 MB &
      CablenetTV &
      \$750 &
      \multicolumn{1}{c|}{Colonia Belgrano-Prov. De Santa Fe}
      \bigstrut\\
    \hline
    \textbf{10 MB+Telefonia+Cable} &
      Canal 4 Pergamino &
      \textbf{\$1.700} &
      \multicolumn{1}{c|}{Pergamino-Provincia de Bs.As}
      \bigstrut\\
    \hline
    \textbf{10 MB+Telefonia+Cable} &
      Conectar Escobar &
      \textbf{\$1.920} &
      \multicolumn{1}{c|}{Escobar-Provincia de Bs As.}
      \bigstrut\\
    \hline
    4 Mb &
      Megacable &
      \$840 &
      \multicolumn{1}{c|}{San Luis Capital}
      \bigstrut\\
    \hline
    6Mb &
      Megacable &
      \$1.040 &
      \multicolumn{1}{c|}{San Luis Capital}
      \bigstrut\\
    \hline
    8Mb &
      Megacable &
      \$1.150 &
      \multicolumn{1}{c|}{San Luis Capital}
      \bigstrut\\
    \hline
    6Mb &
      CTV &
      \$785 &
      \multicolumn{1}{c|}{San Luis Capital}
      \bigstrut\\
    \hline
    10Mb &
      CTV &
      \$945 &
      \multicolumn{1}{c|}{San Luis Capital}
      \bigstrut\\
    \hline
    6Mb &
      Velocom &
      \$1.079 &
      \multicolumn{1}{c|}{Ciudad de Mendoza}
      \bigstrut\\
    \hline
    10Mb &
      Velocom &
      \$1.603 &
      \multicolumn{1}{c|}{Ciudad de Mendoza}
      \bigstrut\\
    \hline
    4Mb &
      Cooperativa Merlo &
      \$1.152 &
      \multicolumn{1}{c|}{Merlo-Provincia de San Luis}
      \bigstrut\\
    \hline
    \textbf{4 Mb + Cable} &
      Cooperativa Merlo &
      \textbf{\$1.683} &
      \multicolumn{1}{c|}{Merlo-Provincia de San Luis}
      \bigstrut\\
    \hline
    \rowcolor[rgb]{ .859,  .898,  .945} \multicolumn{2}{|c|}{\textcolor[rgb]{ .122,  .286,  .49}{\textbf{Promedio Total de Planes Seleccionados}}} &
      \textbf{\$1.627.6} &
      \cellcolor[rgb]{ 1,  1,  1}
      \bigstrut\\
\cline{1-3}    \end{tabular}%
  \caption{Planes de referencia.}
  \label{tab:planes}%
\end{table}%






Considerando el valor promedio obtenido a través de la recolección de los precios de planes similares podemos deducir la proporción destinada a las TIC's por parte de una familia establecida en la Localidad de Papagayos se determina a través del siguiente cálculo:

\begin{equation}
\dfrac{Costo \, Promedio\,TIC's}{Aporte\,medio\,3\, personas}=\dfrac{\$1627,6}{\$36610,55} \simeq 0,045
\end{equation}

Por lo tanto el análisis preliminar muestra que el porcentaje asociado al gasto de las TIC's corresponde al 4,5\%.













%bib3: https://calculadoradeinflacion.com/argentina.html?md=agosto&ad=2018&mh=septiembre&ah=2019&q=31400

%bib2: https://www.cba.gov.ar/la-jubilacion-media-provincial-supera-los-50-500/

%bib3: : https://www.lovemondays.com.ar/

%bib4: Apendice A compartido en drive: https://drive.google.com/drive/folders/1vu527bN0kaPhcQiW2mn9SPwRCseGwazx?usp=sharing





%---------------comienzo de bibliografia-------------------
\begin{thebibliography}{99}


\bibitem{base-datos}Base de datos sobre sueldos:{\tiny  \url{https://www.lovemondays.com.ar/}}

\bibitem{jub-media}ANSES Datos Abiertos. Junio 2019: {\tiny \url{https://www.anses.gob.ar/archivo/p12}}

Archivo en formato xlsx. Celda B83.

\bibitem{inflacion}Cálculo de inflación realizado: {\tiny \url{https://calculadoradeinflacion.com/argentina.html?md=junio&ad=2019&mh=noviembre&ah=2020&q=18075}}

\bibitem{drive}Apendice A compartido en drive: {\tiny \url{https://drive.google.com/drive/folders/1vu527bN0kaPhcQiW2mn9SPwRCseGwazx?usp=sharing}}

\end{thebibliography}