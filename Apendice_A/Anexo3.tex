\section{Anexo C: Flujo de Caja del Negocio}
La proyección del flujo de caja se evalúa a través de la información recopilada a lo largo del informe. Gracias al mismo es posible evaluar la posible rentabilidad del negocio.

\subsection{Estructura}
La estructura del Flujo se basa en la siguiente Tabla:

% Table generated by Excel2LaTeX from sheet 'Hoja1'
\begin{table}[H]
  \centering
    \begin{tabular}{|c|}
    \hline
    \textcolor[rgb]{ 0,  .69,  .314}{+ Ingresos afectos a impuestos} \bigstrut\\
    \hline
    \textcolor[rgb]{ 1,  0,  0}{- Egresos afectos a impuestos} \bigstrut\\
    \hline
    \textcolor[rgb]{ 1,  0,  0}{- Intereses del préstamo} \bigstrut\\
    \hline
    \textcolor[rgb]{ 1,  0,  0}{- Gastos no desembolsables} \bigstrut\\
    \hline
    \textcolor[rgb]{ 0,  .439,  .753}{= Utilidad antes del impuesto} \bigstrut\\
    \hline
    \textcolor[rgb]{ 1,  0,  0}{- Impuesto} \bigstrut\\
    \hline
    \textcolor[rgb]{ 0,  .69,  .314}{+ Ajustes por gastos no desembolsables} \bigstrut\\
    \hline
    \textcolor[rgb]{ 1,  0,  0}{- Egresos no afectos a impuestos} \bigstrut\\
    \hline
    \textcolor[rgb]{ 0,  .69,  .314}{+ Beneficios no afectos a impuestos} \bigstrut\\
    \hline
    \textcolor[rgb]{ 0,  .69,  .314}{+ Préstamo} \bigstrut\\
    \hline
    \textcolor[rgb]{ 1,  0,  0}{- Amortización de la deuda} \bigstrut\\
    \hline
    \textcolor[rgb]{ 0,  .439,  .753}{= Flujo de Caja} \bigstrut\\
    \hline
    \end{tabular}%
      \caption{Estructura general del Flujo de Caja.}
  \label{tab:estructura}%
\end{table}%

\begin{enumerate}
\item[•]\textbf{Ingresos afectos a impuestos}:

Están constituidos por los ingresos esperados por la 
venta de los productos, lo que se calcula multiplicando el precio de cada unidad por 
la cantidad de unidades que se proyecta producir y vender cada año.

Los valores incluidos en esta categoría son los siguientes:
\begin{enumerate}
\item[•]Abonados al plan bronce.

Calculado a través de la siguiente ecuación:
\begin{equation}
N_{abonados-plan-bronce}\cdot Precio_{plan-bronce}=Ingreso_{plan-bronce}
\end{equation}

\item[•]Abonados al plan plata.

Calculado a través de la siguiente ecuación:
\begin{equation}
N_{abonados-plan-plata}\cdot Precio_{plan-plata}=Ingreso_{plan-plata}
\end{equation}

\item[•]Abonados al plan oro.

Calculado a través de la siguiente ecuación:
\begin{equation}
N_{abonados-plan-oro}\cdot Precio_{plan-oro}=Ingreso_{plan-oro}
\end{equation}

\item[•]Ingresos por publicidad.

Calculado a través de la tarifa mostrada en el Capítulo VIII del informe. Considerando que las publicidades del tipo
rotativo comercial, digitado comercial mañana y publicidad política se repiten hasta 4 veces por día.

\end{enumerate}

\item[•]\textbf{Egresos afectos a impuestos}:
Corresponden a los costos variables resultantes del 
costo de fabricación unitario por las unidades producidas, el costo anual fijo.

Los valores incluidos en esta categoría son los siguientes:
\begin{enumerate}
\item[•]Costo del servicio de internet mayorista.

Calculado a través de la siguiente ecuación:
\begin{equation}
Costo_{mensual}\cdot 12 = Costo_{anual}
\end{equation}

\item[•]Costo del servicio de SDTV,HDTV: Conseguido a través del pago de un canon para acceder al Hub de Contenidos del CABASE.
El cual es un servicio disponible a los proveedores ISP que se encuentran asociados a algún IXP.
%Fuente: https://www.cabase.org.ar/wp-content/uploads/2020/01/Folleto-Hub-de-Contenidos_c-1.pdf
Calculado a partír de la siguiente ecuación:

\begin{equation}
Costo_{mensual-SDTV-HDTV}\cdot 12 = Costo_{anual-SDTV-HDTV}
\end{equation}

\item[•]VoIP mayorista: Calculado a partír del costo del servicio mayorista de internet.
Calculado a partir de la siguiente ecuación:

\begin{equation}
Costo_{mensual-VoIP}\cdot 12 = Costo_{anual-VoIP}
\end{equation}

\item[•]Equipos y accesorios de telecomunicaciones: Este item se encuentra compuesto por los precios de los equipos y accesorios 
descontando los impuestos que se aplican, tales como IVA, y los impuestos de Aduana. El monto se calcula a partir de la suma
de los precios descontados de los equipos mencionados en las secciones del Plan Económico de Equipamiento de la red de Distribución
y Acceso por Fibra Óptica.

\item[•]Costo de Nuevas Conexiones: Se lo define como el gasto en el que la empresa adquiere cuando realiza una nueva conexión al servicio.
%Se estima que en el futuro la expansión del numero de abonados sera la siguiente:
Según un informe del Observatorio Permanente de la Industria del Software y Servicios Informáticos de la República
(OPSSI), en febrero de 2019 se espera una evolución de las ventas anual de alrededor del 25\%, 
en este caso se considera que dicha evolución se establece en un plazo de 5 años.
A partir de esta etapa, se supone que el servicio se encuentra en su etapa de crecimiento pleno, de forma tal que el número de
abonados al servicio incrementa un 6\% anual.

\begin{enumerate}
\item[•]1$^{\circ}$Año:61 clientes.
\item[•]2$^{\circ}$Año:61 clientes.
\item[•]3$^{\circ}$Año:61 clientes.
\item[•]4$^{\circ}$Año:61 clientes.
\item[•]5$^{\circ}$Año:61 clientes.
\item[•]6$^{\circ}$Año:65 clientes.
\item[•]7$^{\circ}$Año:69 clientes.
\item[•]8$^{\circ}$Año:74 clientes.
\item[•]9$^{\circ}$Año:79 clientes.
\item[•]10$^{\circ}$Año:84 clientes.
\end{enumerate}





\end{enumerate}



\item[•]\textbf{Intereses del préstamo}:
Monto de las cuotas abonadas al préstamo realizado por la ENACOM cuyo monto corresponde a US\$ 52000, en la modalidad UVA (Unidad de Valor Adquisitivo), con una tasa fija del 2\% .

%https://www.enacom.gob.ar/SU/programa-creditos-preferenciales_sup5
%https://www.argentina.gob.ar/sites/default/files/infoleg/res548-02.pdf

%\item[•]\textbf{Gastos no desembolsables}:
%Están compuestos por la depreciación, la amortización de intangibles.
%
%Los valores incluidos en esta categoría son los siguientes:
%\begin{enumerate}
%\item[•]
%\item[•]
%\item[•]
%\end{enumerate}




\item[•]\textbf{Impuestos}:

Los valores incluidos en esta categoría son los siguientes:
\begin{enumerate}
\item[•]Con respecto a la compra de equipos importados:
\begin{enumerate}
\item[•]IVA Tasa General Res. AFIP 3373/2012.
\item[•]IVA Adicional Res. AFIP 3373/2012.
\item[•]Ingresos Brutos Res. AFIP 3373/2012.
\item[•]Tasa Digitalización.
\item[•]Tasa Oficialización.
\item[•]Impuesto a las ganancias AFIP 3373/2012.
\end{enumerate}

\item[•]Con respecto al pago del IVA de posiciones mensuales (El calculo del monto se ilustra en la seccion inferior de la Tabla de Flujos).

\end{enumerate}



%\item[•]\textbf{Ajuste por gastos no desembolsables}:
%Se suman la depreciación, la amortización 
%de intangibles para anular el efecto de haber incluido gastos que no constituían egresos de caja.
%
%Los valores incluidos en esta categoría son los siguientes:
%\begin{enumerate}
%\item[•]
%\item[•]
%\item[•]
%\end{enumerate}


%\item[•]\textbf{Egresos no afectos a impuestos}:
%Ente los egresos no afectos a impuestos se encuentran las inversiones, ya que no aumentan ni disminuyen la riqueza contable de la empresa por el solo hecho de adquirirlos.
%
%Los valores incluidos en esta categoría son los siguientes:
%\begin{enumerate}
%%##########      PARA NOSOTROS ESTO VA EN LA PARTE DE EGRESOS NO AFECTOS A IMPUESTOS
%\item[•]Contratación Personal: Se encuentra incluido el capital invertido para la construcción de la red. Incluye los salarios de los técnicos, ingenieros
%y personal administrativo. El valor de los mismos se encuentra detallado en el Plan de Acción.
%
%%Preguntar al profe
%
%
%\item[•] 
%\item[•]
%\end{enumerate}


%\item[•]\textbf{Beneficios no afectos a impuestos}:
%Son el valor de desecho del proyecto y la recuperación del capital de trabajo. No se toma en cuenta en este informe.



\end{enumerate}