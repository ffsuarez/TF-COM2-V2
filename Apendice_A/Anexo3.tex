\section{Apéndice Flujo de Caja del Negocio}
La proyección del flujo de caja se evalúa a través de la información recopilada a lo largo del informe. Gracias al mismo es posible evaluar la posible rentabilidad del negocio.

\subsection{Estructura}
La estructura del Flujo se basa en la siguiente Tabla:

% Table generated by Excel2LaTeX from sheet 'Hoja1'
\begin{table}[H]
  \centering
    \begin{tabular}{|c|}
    \hline
    \textcolor[rgb]{ 0,  .69,  .314}{+ Ingresos afectos a impuestos} \bigstrut\\
    \hline
    \textcolor[rgb]{ 1,  0,  0}{- Egresos afectos a impuestos} \bigstrut\\
    \hline
    \textcolor[rgb]{ 1,  0,  0}{- Intereses del préstamo} \bigstrut\\
    \hline
    \textcolor[rgb]{ 1,  0,  0}{- Gastos no desembolsables} \bigstrut\\
    \hline
    \textcolor[rgb]{ 0,  .439,  .753}{= Utilidad antes del impuesto} \bigstrut\\
    \hline
    \textcolor[rgb]{ 1,  0,  0}{- Impuesto} \bigstrut\\
    \hline
    \textcolor[rgb]{ 0,  .69,  .314}{+ Ajustes por gastos no desembolsables} \bigstrut\\
    \hline
    \textcolor[rgb]{ 1,  0,  0}{- Egresos no afectos a impuestos} \bigstrut\\
    \hline
    \textcolor[rgb]{ 0,  .69,  .314}{+ Beneficios no afectos a impuestos} \bigstrut\\
    \hline
    \textcolor[rgb]{ 0,  .69,  .314}{+ Préstamo} \bigstrut\\
    \hline
    \textcolor[rgb]{ 1,  0,  0}{- Amortización de la deuda} \bigstrut\\
    \hline
    \textcolor[rgb]{ 0,  .439,  .753}{= Flujo de Caja} \bigstrut\\
    \hline
    \end{tabular}%
      \caption{Estructura general del Flujo de Caja.}
  \label{tab:estructura}%
\end{table}%

\begin{enumerate}
\item[•]\textbf{Ingresos afectos a impuestos}:

Están constituidos por los ingresos esperados por la 
venta de los productos, lo que se calcula multiplicando el precio de cada unidad por 
la cantidad de unidades que se proyecta producir y vender cada año.

Los valores incluidos en esta categoría son los siguientes:
\begin{enumerate}
\item[•]
\item[•]
\item[•]
\end{enumerate}

\item[•]\textbf{Egresos afectos a impuestos}:
Corresponden a los costos variables resultantes del 
costo de fabricación unitario por las unidades producidas, el costo anual fijo.

Los valores incluidos en esta categoría son los siguientes:
\begin{enumerate}
\item[•]
\item[•]
\item[•]
\end{enumerate}



\item[•]\textbf{Intereses del préstamo}:
Monto de las cuotas y la composición de ellas entre intereses y amortización.

Los valores incluidos en esta categoría son los siguientes:
\begin{enumerate}
\item[•]
\item[•]
\item[•]
\end{enumerate}



\item[•]\textbf{Gastos no desembolsables}:
Están compuestos por la depreciación, la amortización de intangibles.

Los valores incluidos en esta categoría son los siguientes:
\begin{enumerate}
\item[•]
\item[•]
\item[•]
\end{enumerate}




\item[•]\textbf{Impuesto}:
Se determina como el \% de las utilidades antes del impuesto.

Los valores incluidos en esta categoría son los siguientes:
\begin{enumerate}
\item[•]
\item[•]
\item[•]
\end{enumerate}



\item[•]\textbf{Ajuste por gastos no desembolsables}:
Se suman la depreciación, la amortización 
de intangibles para anular el efecto de haber incluido gas-tos que no constituían egresos de caja.

Los valores incluidos en esta categoría son los siguientes:
\begin{enumerate}
\item[•]
\item[•]
\item[•]
\end{enumerate}


\item[•]\textbf{Egresos no afectos a impuestos}:
Ente los egresos no afectos a impuestos se encuentran las inversiones, ya que no aumentan ni disminuyen la riqueza contable de la empresa por el solo hecho de adquirirlos.

Los valores incluidos en esta categoría son los siguientes:
\begin{enumerate}
\item[•]
\item[•]
\item[•]
\end{enumerate}


\item[•]\textbf{Beneficios no afectos a impuestos}:
Son el valor de desecho del proyecto y la recuperación del capital de trabajo. No se toma en cuenta en este informe.



\end{enumerate}