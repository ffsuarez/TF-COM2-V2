% A lo largo del informe se desarrolla de forma hipotética la posibilidad de implementación de una red de telecomunicaciones entre Carpintería y Papagayos, considerando un servicio de voz, video y datos. Se incluyen distintas alternativas y los costos a los que debe enfrentarse una organización que lleva a cabo esta tarea.

% Bajo los análisis que se realizan se determina la gran dificultad económica que se enfrenta una organización que desea lucrar a través de esta actividad, principalmente debido al bajo volumen de ingresos y el gran costo para la implementación de la red.

% Es posible advertir que la alternativa que es más barata constituye a su vez la que tiene menos posibilidad de extensión del mercado, lo cual consideramos como un factor crucial para la redituabilidad del proyecto.

% %Durante la confección del informe fue posible incorporar los conocimientos que se aprendieron durante el cursado de la materia en pos de lograr la instalación mas conveniente de acuerdo a los criterios de los integrantes del equipo. También fue posible visualizar situaciones y conceptos que juegan un papel preponderante durante la instalación de la red. 

% Para el diseño de la red es preciso realizar un estudio teórico sobre las características de la fibra óptica, para poder comprender la tecnología en la que se basa y así ofrecer una solución final acorde a las necesidades que se presenten. 

% %De esta forma, se concretan las características y el funcionamiento de diversos elementos, tanto activos como pasivos, que intervienen en una red de acceso de fibra
% %óptica. Dichos elementos son necesarios para la implementación de la red y asegurar su fiabilidad de manera tal de dotarla de maniobrabilidad para extenderla o ampliarla hacia en la misma ciudad o otras localidades.

% Tras un vistazo general a la tecnología de fibra óptica, se procede a estudiar las características y topologías de las redes, junto con los requisitos de cara al diseño y a la gestión de proyectos. Dicho estudio abarca desde estudios
% de mercado, cuestiones legales, pasos en las fases de diseño, elección de materiales y replanteo. 

% %Antes de proceder al diseño de una red de fibra óptica, se considera oportuno explicar con detalle ciertos aspectos prácticos del proyecto: como los criterios de
% %instalación de fibra o las técnicas de despliegue de la misma, según por el medio que atraviesen.

% %Con todo lo anterior, ya se dispone de información suficiente para el diseño de la red. 
% Con todo lo anterior se recrea un escenario basado en la región de despliegue para analizar la viabilidad de la red. El resultado de este análisis indica que no es rentable hasta que transcurran 10 años, por lo tanto se considera un proyecto inviable.

% En ocasiones, los operadores muestran preocupados para desplegar redes de fibra óptica debido al desembolso inicial que implica el despliegue de una red de estas características. No obstante el alquiler de la red a otros operadores genera unos beneficios que compensa este desembolso inicial y alienta a las grandes compañías a
% invertir en esta tecnología.

% Gracias a este proyecto y a otros estudios, se ha demostrado que la tecnología de fibra óptica es apta para las crecientes demandas de servicios como internet de banda ancha, o televisión digital en alta definición. La fibra óptica tiene mucho que ofrecer a las telecomunicaciones, y cada vez es más habitual encontrar redes de fibra óptica como parte de un servicio de telecomunicaciones, tanto como para particulares como empresas. 


% % Colocar cosas

% Por otro lado, la realización de dicho proyecto fue muy enriquecedor en cuanto a la clase de investigaciones realizadas, manejándonos en campos totalmente desconocidos en cuanto a nuestros conocimientos. 

% Se obtuvo un gran aprendizaje en la forma de abordar los estudios socioeconomicos a partir de censos, noticias y especulaciones propias basadas en lo anteriormente mencionado.


% También se adquirió una gran formación en la realización de enfoques, análisis y especulaciones sobre diversos temas, tales como  poder adquisitivo, las tendencias de los habitantes de la región, etc. 

% Se aprendió a clasificar la distinta información que se encuentra a nuestra disposición y solo tomar aquella proveniente de fuentes confiables y comprobables, ya que en Internet circula un gran volumen de información el cual no es del todo cierta.
 
% Se logró cierto nivel de experiencia en cuanto al contacto con proveedores de equipos y servicios, así como también con autoridades municipales con el fin de obtener información real y actual sobre las características que se consideraron relevantes de las localidades en estudio.

% En cuanto al trabajo en equipo, el aborde de dicho proyecto fue muy beneficioso. Trabajar de esta forma permitió compartir ideas, nuevos conocimientos y distintos puntos de vista, siempre manteniendo el respeto y aprendiendo a manejar las diferencias que surgen entre los integrantes del equipo siempre de manera respetuosa.

% Pudimos notar que la ingeniería va mucho mas allá de lo que aparece en los libros. La gran mayoría de las decisiones tomadas, en base de toda la información recolectada, fue por el sentido común, algo muy inculcado durante toda la cursada de la materia.

% % Fin 














La posibilidad de implementación de la red de telecomunicaciones entre Carpintería
y Papagayos se ha analizado desde diversos puntos de vista a partír de la información
que se encuentra a disposición. 

Desde un aspecto demográfico es posible advertir que el pequeño número de habitantes que 
pueden ser abonados potenciales y el monto que una familia de la región
invierte en las TIC's impacta de forma negativa a la posibilidad de rentabilidad
económica. 

Desde un aspecto tecnico se desarrollan las alternativas de mayor conveniencia, y 
finalmente se decide que en la red de Distribución 
es más adecuada la colocación de fibra óptica debido a que presenta mayor
fiabilidad, mayor capacidad de satisfacción de la demanda, una firme posibilidad de expansión
hacia mercados de mayor volumen y el costo económico de adquisición es similar a la de una 
red por radioenlace. 

Mientras que en la Red de Acceso se decide el montaje de una red GPON 
por razones similares a las anteriores, se destaca que a pesar de que el tiempo de
implementación sea más lento, el costo final de implementación
es levemente inferior a la alternativa inalámbrica, la pequeña extensión de la Localidad ofrece
mayor simpleza durante las actividades de mantenimiento, y su vigencia tecnológica se encuentra
asegurada por mucho tiempo.

Desde un punto de vista económico y financiero, se ha realizado una proyección del flujo
de fondos a partír de la información contenida en los análisis anteriores, de manera que
al efectuar la evaluación a través del cálculo de Valor Actual Neto y Tasa Interna de Retorno
se concluye que no se experimenta un retorno de la inversión 
hasta que transcurran 6 años, lo cual implica que a efectos prácticos este Plan de Negocios no
es rentable.
