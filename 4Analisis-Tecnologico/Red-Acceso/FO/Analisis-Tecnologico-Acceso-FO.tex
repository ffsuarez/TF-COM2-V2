\section{Red de Acceso por Fibra Óptica}

Se plantea la instalación de una red de acceso óptica, cuya topología se encuentra caracterizada por la presencia de un nodo de acceso ubicado sobre la terminación del enlace troncal.

Existen diversas tecnologías para una red de acceso por fibra óptica, una de ellas es FTTH (Fiber To The Home), diseñada específicamente para la transmisión de voz, video y datos. 
En la Figura \ref{dem-figuras:FTTH} se puede apreciar diferentes soluciones de FTTH y sus características distintivas.
\FigureDEM{H}{10}{FTTH}{Soluciones FTTH.}
De entre todas ellas se decide utilizar una red óptica pasiva (PON), dado que asegura la tasa demandada por cada hogar al tiempo que minimiza los costos económicos al utilizar solamente splitters pasivos. 

Entre las diversas tecnologías PON se encuentra el estándar GPON (Gigabit capable Passive Optical Network). El cual permite aplicar una red de acceso punto-multipunto, donde se decide utilizar la versión que soporta hasta 128 usuarios por puerto ,alcanzando tasas teóricas de hasta 2,4 Gbps en downstream y 1,2 Gbps en upstream.
En la Figura \ref{dem-figuras:GPON} se ilustra la topología de la red, junto con los elementos que la componen.

\FigureDEM{H}{10}{GPON}{Topología de red GPON.}

\subsection{Componentes Pasivos}



\begin{enumerate}
\item[•]Fibra Óptica.




\begin{enumerate}
\item[•]El cable (o el grupo de cables) que interconecta el puerto PON con la entrada al splitter primario se denomina \textit{feeder}, para esta aplicación dicho splitter se encuentra instalado sobre un rack, de forma que este tipo de cable no se utiliza.

\item[•]El cable de distribución corresponde a la fibra óptica conectada al splitter secundario, al analizar la ubicación de las cajas terminales de distribución, se plantea la instalación de este tipo de cables a través de los postes de energía.
Se utiliza la misma fibra óptica seleccionada para el enlace troncal, en la Tabla \ref{tab:fo-analizadas} se describen sus características más destacadas.

\item[•]El cable de \textit{drop} corresponde a la fibra óptica que se tiende desde la salida del splitter secundario, hacia la roseta óptica ubicada en el hogar del abonado.
El cable de fibra óptica seleccionado para esta sección de la red, corresponde TW-SCIE del fabricante OEM. La Tabla \ref{tab:fo-drop} ilustra sus características mas destacadas.


%    \textcolor[rgb]{ .004,  .004,  .004}{TW-SCIE} & OEM   & Monomodo & $\leq$ 0,4 (1310nm) $\leq$0,3 (1550nm) & $\leq$ 0,5 (1550nm, 7,5mm x 1 vuelta) & 0,035 \bigstrut\\

% Table generated by Excel2LaTeX from sheet 'Hoja2'
\begin{table}[H]
  \tiny
  \centering
    \begin{tabular}{|l|c|c|c|c|c|}
    \hline
    \rowcolor[rgb]{ .773,  .851,  .945} \multicolumn{1}{|c|}{\textbf{Modelo}} & \textbf{Fabricante} & \textbf{Nro de fibras} & \textbf{Atenuacion en dB/km} & \textbf{Perdida por curvatura en dB} & \textbf{Costo US\$} \bigstrut\\
    \hline
        \textcolor[rgb]{ .004,  .004,  .004}{TW-SCIE} & OEM   & 2 & $\leq$ 0,4 (1310nm) $\leq$0,3 (1550nm) & $\leq$ 0,5 (1550nm, 7,5mm x 1 vuelta) & 0,035 \bigstrut\\
    \hline
    \end{tabular}%
	\caption{Características Fibra Óptica drop.}
  \label{tab:fo-drop}%
\end{table}%





\end{enumerate}


\item[•]Empalmes.
Su propósito es el de aumentar el alcanze de la fibra óptica. Existen dos metodologías principales, los empalmes mecánicos y por fusión. Esta última metodología es la que se utiliza. El cálculo de la perdida de presupuesto óptico, el cual considera las pérdidas producidas, entre otros aspectos, por los empalmes en el enlace troncal, se encuentran detallados en el Apéndice B.


\item[•]Splitters.

Dividen la señal óptica en tantos caminos como indique su relación. Pueden ser instalados de distintas formas:
\begin{enumerate}
\item[•]En armarios ópticos: También llamados tipo ''cassette". Este tipo de splitter se utiliza en el primer nivel de splitter.
\item[•]Subterráneos: Enterrados de forma tal que no están expuestos a manipulaciones.
\item[•]Aéreos: Instalados en una caja de splitter o una caja terminal montada sobre una columna o adosada a una pared. La fibra de entrada se empalma dentro de la caja terminal, mientras que la fibra de salida puede empalmarse directamente al cable de acometida del abonado o puede utilizar conectores. Este tipo se utiliza en el último nivel de splitter. También son denominados ''PLC".

\end{enumerate}




\item[•]Conectores.


Los conectores más comunes usados en la fibra óptica para redes de área local son los conectores ST, LC, FC Y SC. En este proyecto se considera el conector del tipo SC (Suscriptor Connector) el cual es un conector de broche y esta estandarizado en TIA-568-A( Específica los requerimientos mínimos para el cableado de 
establecimientos)

Dentro de los conectores SC se puede clasificar de acuerdo al pulido que poseen:

\begin{itemize}
\item PC: Contacto Físico (Physical Contact), el pin está biselado y rematado en una superficie plana. Esto evita espacios vacíos entre los pines de los conectores que están acoplados y logra unas pérdidas de retorno entre los -30 dB y los -40 dB. 
\item UPC: Ultra Contacto Físico (Ultra Physical Contact), similares a los PC, pero logran reducir las pérdidas de retorno a un margen entre los -40 y los -55 dB gracias a que el bisel tiene una curva más pronunciada.
\item APC: Contacto Físico en Ángulo (Angled Physical Contact), el pin termina en una superficie plana y a su vez inclinada 8 grados. Se trata del conector que logra un enlace óptico de mayor calidad ya que consigue reducir las pérdidas de retorno hasta los -60 dB aumentando así el número de usuarios en fibras monomodo. Este conector es el seleccionado para implementar en la red.
\end{itemize}


\FigureDEM{H}{12}{ddj}{Tipos de conectores SC.}



Los conectores SC/APC se utilizan para:
\begin{itemize}
\item Conexión en el enlace troncal.
\item Conexión entre splitters.
\item Conexión de clientes a splitters.
\end{itemize}   



\item[•]Armarios/Cajas de Empalme/Cabinas Exteriores.

Los armarios de distribución permiten alojar splitters del tipo modular (''cassette") o también del tipo PLC.

La Figura \ref{dem-figuras:caja_distribucion} ilustra la disposición del armario de distribución dentro de la red. En la misma, se ilustra con recuadro de color rosa, la disposición dentro de la tecnología FTTH.

\FigureDEM{H}{16}{caja_distribucion}{Disposición de caja de distribución en red acceso.}



En la Figura \ref{dem-figuras:figA1} se delimita la zona a analizar.

\FigureDEM{H}{12}{figA1}{Ciudad de Papagayos.}

Se diagrama una repartición geográfica de los hogares, con el propósito de determinar la ubicación de las cajas de distribución. Sobre la Figura \ref{dem-figuras:figA2} se ilustra la subdivisión de la Localidad en diversas áreas, es posible determinar zonas de baja, media y alta dispersión, los cuales permiten delimitar la forma más conveniente de implementar los equipos para la distribución del servicio.


Al considerar el caso en que se elija un enlace troncal por fibra óptica como en la sección anterior, la OLT se observa en la Figura \ref{dem-figuras:figA2}(marcada en color naranja). 

\FigureDEM{H}{16}{figA2}{Subdivisión de la ciudad en áreas.}



Se utilizan 2 puertos en la OLT, en ambos se utiliza un patchcord conectado a un splitter de 1:4 del tipo ''Cassette", teniendo en cuenta que uno de ellos permanece desconectado ante una eventual falla. En el montaje de los  terminales de acceso de fibra (FAT) se coloca un splitter 1:32 del tipo ''PLC"  para iniciar el recubrimiento del área donde se encuentran dispersos los clientes, la Figura \ref{dem-figuras:figE1} determina la ubicación de cada uno de estos terminales de acceso.


\FigureDEM{H}{15}{figE1}{Ubicaciones de terminales de acceso FAT.}







\subsection{Componentes Activos}
En tanto que los elementos activos específicos de la red corresponden a:
\begin{enumerate}
\item[•]Terminación de Linea Óptica (OLT).

Responsable de la interconexión de la red de acceso con el enlace troncal del operador. Gestiona,administra y sincroniza el tráfico con los equipos terminales (ONT's) en modalidad TDMA de manera que tiene un solo medio compartido por donde viajan los datos sincronizados en uplink y downlink de todas las ONT's asociadas a este equipo.

Debido a que se tiene la transmisión de los datos en ambos sentidos, se utilizan dos longitudes de onda para diferenciar el flujo de datos, siendo los valores usados 1310 nm en upstream y 1550 nm en downstream.




El análisis para verificar si la red soporta todo el tráfico demandado parte de la suposición de que se utiliza el estándar que permite hasta 1,2 Gbps de downstream.


Existen algunas limitaciones que inciden sobre el número ONT’s que se utilizan para la distribución del servicio:
\begin{itemize}
\item El estándar GPON impone un límite de 64 clientes que puede sincronizar por puerto de la OLT, en particular, las últimas especificaciones permiten hasta 128 usuarios. Está última especificación es la que se utiliza.
\item La tasa ocupada por cada cliente impone una restricción sobre el número de clientes a partir de la velocidad de transmisión aplicada por el protocolo GPON.
\item La distribución de los clientes también impone una restricción, dado que no resulta conveniente el montaje de un cable de grandes longitudes para alcanzar clientes repartidos en un área extensa. Sin embargo, a pesar de que no se encuentra recomendado, se estima que en muy pocos hogares se presenta este problema.
\end{itemize}











Sabiendo la capacidad de una OLT en GPON en ambos sentidos y la tasa de transferencia por hogar máxima posible calculada en la Ecuación \ref{m3}, se deduce que un puerto OLT puede asociar la cantidad de usuarios indicada en la Ecuación \ref{m4}.

\begin{equation}
C_{MAX}=\frac{TASA_{MAX}}{TASA_{MAX\hspace{0.15cm}POR\hspace{0.15cm}HOGAR}}=\frac{1.25Gbps}{15Mbps}=83\hspace{0.15cm}usuarios
\label{m4}
\end{equation}

Por lo tanto con la utilización de 1 solo puerto OLT es posible abastecer a los 61 usuarios abonados al servicio.


Para la elección de la OLT a utilizar, se presentan dos opciones de distintos fabricantes, cuyas especificaciones generales están en la Tabla \ref{tab:caracteristicas-olt}.


% Table generated by Excel2LaTeX from sheet 'Hoja3'
\begin{table}[H]
  \tiny
  \centering  
    \begin{tabular}{|c|c|c|c|p{5.335em}|p{7.665em}|c|}
    \hline
    \rowcolor[rgb]{ .773,  .851,  .945} \textbf{Equipo} & \textbf{Fabricante} & \textbf{TX (dBm)} & \textbf{RX (dBm)} & \multicolumn{1}{c|}{\textbf{Interfaces}} & \multicolumn{1}{c|}{\textbf{GPON Speeds}} & \textbf{Costo (US\$)} \bigstrut\\
    \hline
    UF-OLT-4 & Ubiquiti & 1.5  to 5  & -28 dBm to -8 dBm & (4) GPON OLT SFP\newline{}(1) 1G/10G SFP+ & 2.488 Gbps Upstream (TX)\newline{}1.244 Gbps Downstream (RX) & 1098 \bigstrut\\
    \hline
    Lucent 7342 ISAM FTTU & Alcatel & 5     & -28 dBm to -8 dBm & (2) GPON OLT \newline{}(12) Ethernet  & 2.488 Gbps Upstream (TX)\newline{}1.244 Gbps Downstream (RX) & 1000 \bigstrut\\
    \hline
    \end{tabular}%
	\caption{Características de OLT's.}
  \label{tab:caracteristicas-olt}%
\end{table}%



Para realizar la elección de los equipos se plantean los siguientes items:

\begin{itemize}
\item \textbf{ Información ofrecida:} Alcatel ofrece una considerable cantidad de información acerca de los recursos que se pueden utilizar para la mejora de la calidad del servicio al cliente.
\item \textbf{ Alcance máximo de la red  óptica:} El alcance máximo calculado para ambos equipos es el mismo tanto para el Upstream como el Downstream, por lo que no representa un factor determinante para la elección del equipo.
\item \textbf{Eficiencia en la utilización de los recursos:} La OLT ofrecida por Ubiquiti posee 4 puertos, donde cada puerto otorga servicio a 128 ONT. La OLT ofrecida por Alcatel posee 8 puertos donde cada puerto otorga servicio a 64 ONT. Con el equipo Ubiquiti, es posible abastecer a todas las ONT potenciales calculadas con un único puerto, quedando solo tres puertos libres, mientras que para la ofrecida por Alcatel, se necesitan 2 puertos para la misma cantidad de ONT, quedando 6 puertos libres. Ubiquiti presenta una mejor eficiencia en cuanto a la utilización de los recursos del equipo.
\item \textbf{Antigüedad de equipos:} Debido al año de emisión del datasheet y que solo se encuentran a la venta equipos usados de Alcatel, se deduce que se trata de un equipo antiguo y puede no presentar soporte técnico, presentando una gran desventaja. 
\end{itemize}



Con lo mencionado anteriormente, se escoge el equipo de Ubiquiti para la implementación de la red  óptica.






\item[•]Terminación de Red Óptica (ONT).

Dispositivo ubicado sobre el hogar del abonado y permite la conexión del mismo sobre la red óptica, sobre la Figura \ref{dem-figuras:caja_distribucion} se ilustra su disposición dentro de la red , para así acceder a los servicios \textit{Triple Play}.

Para la elección de la ONT se tiene en cuenta los equipos mostrados en la Tabla \ref{tab:caracteristicas-ont}.


% Table generated by Excel2LaTeX from sheet 'Hoja1'
\begin{table}[H]
  \tiny
  \centering
    \begin{tabular}{|c|c|c|c|p{15.61em}|p{12.5em}|c|}
    \hline
    \rowcolor[rgb]{ .773,  .851,  .945} \textbf{Equipo} & \textbf{Fabricante} & \textbf{TX (dBm)} & \textbf{RX (dBm)} & \multicolumn{1}{c|}{\textbf{Interfaces}} & \multicolumn{1}{c|}{\textbf{GPON Speeds}} & \textbf{Costo (US\$)} \bigstrut\\
    \hline
    UF-LOCO & Ubiquiti & 1.5 to 5 dBm & -28 to -8 dBm & (1) SC/APC, GPON WAN\newline{}(1) Gigabit RJ45, Ethernet LAN\newline{} & 2.488 Gbps Downstream\newline{}1.244 Gbps Upstream & 89 \bigstrut\\
    \hline
    HG8247H & Huawei & 0.5 to 5 dBm & -27  to -8 dBm & (1) SC/APC, GPON WAN\newline{}(4) Gigabit RJ45, Ethernet LAN\newline{}(1) CATV\newline{}(2) RJ11, POTS\newline{}WIFI\newline{} & 2.488 Gbps Downstream\newline{}1.244 Gbps Upstream & 54 \bigstrut\\
    \hline
    \end{tabular}%
	\caption{Características ONT's.}
  \label{tab:caracteristicas-ont}%
\end{table}%

Para realizar la elección entre los equipos de Huawei y Ubiquiti, se plantean los siguientes items:

\begin{itemize}
\item Información ofrecida: ambas empresas ofrece una considerable cantidad de información acerca de los recursos que se pueden utilizar para la mejora de la calidad del servicio al cliente.
\item Eficiencia en la utilización de los recursos: La ONT ofrecida por Ubiquiti posee 1 solo puerto de Ethernet y para poder brindar servicio Triple Play se necesitan otros equipos. En cambio la ONT Huawei proporciona 4 puertos GE Ethernet, 2 puerto POTS, CATV, WIFI y adicionalmente tiene la función WIFI. 
\item Compatibilidad: Ambas ONT son compatibles con la OLT Ubiquiti.
\end{itemize}
Con lo mencionado anteriormente, se escoge el equipo de Huawei.

En la ONT seleccionada se puede conectar teléfono IP, computadoras y Smart TV a los puertos LAN. Como este equipo posee 2 puertos POTS el usuario podrá conectar más de un teléfono que no sea IP. La conexión inalámbrica que brinda este dispositivo es una ventaja ya que no se necesita otro equipo para hacerlo, al igual que la salida para obtener CATV directamente. 




\end{enumerate}



\newpage






\end{enumerate}