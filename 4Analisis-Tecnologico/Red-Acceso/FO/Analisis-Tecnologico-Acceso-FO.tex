\section{Red de Acceso por Fibra Óptica}

Se plantea la instalación de una red de acceso óptica, cuya topología se encuentra caracterizada por la presencia de un nodo de acceso ubicado sobre la terminación del enlace troncal.

Existen diversas tecnologías para una red de acceso por fibra óptica, una de ellas es FTTH (Fiber To The Home), diseñada específicamente para la transmisión de voz, video y datos. 
En la Figura \ref{dem-figuras:FTTH} se puede apreciar diferentes soluciones de FTTH y sus características distintivas.
\FigureDEM{H}{10}{FTTH}{Soluciones FTTH.}
De entre todas ellas se decide utilizar una red óptica pasiva (PON), dado que asegura la tasa demandada por cada hogar al tiempo que minimiza los costos económicos al utilizar solamente splitters pasivos. 

Entre las diversas tecnologías PON se encuentra el estándar GPON (Gigabit capable Passive Optical Network). El cual permite aplicar una red de acceso punto-multipunto, donde se decide utilizar la versión que soporta hasta 128 usuarios por puerto ,alcanzando tasas teóricas de hasta 2,4 Gbps en downstream y 1,2 Gbps en upstream.
En la Figura \ref{dem-figuras:GPON} se ilustra la topología de la red, junto con los elementos que la componen.

\FigureDEM{H}{10}{GPON}{Topología de red GPON.}

\subsection{Componentes Pasivos}



\begin{enumerate}
\item[•]Fibra Óptica.


\begin{enumerate}
\item[•]El cable (o el grupo de cables) que interconecta el puerto PON con la entrada al splitter primario se denomina \textit{feeder}, para esta aplicación dicho splitter se encuentra instalado sobre un rack, de forma que este tipo de cable no se utiliza.

\item[•]El cable de distribución corresponde a la fibra óptica conectada al splitter secundario, al analizar la ubicación de las cajas terminales de distribución, se plantea la instalación de este tipo de cables a través de los postes de energía.

\item[•]El cable de \textit{drop} corresponde a la fibra óptica que se tiende desde la salida del splitter secundario, hacia la roseta óptica ubicada en el hogar del abonado.
\end{enumerate}
