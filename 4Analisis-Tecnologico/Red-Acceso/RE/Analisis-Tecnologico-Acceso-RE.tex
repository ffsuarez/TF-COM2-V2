\section{Red de Acceso Inalámbrico}
Se considera el montaje de una red inalámbrica de acceso sobre la localidad de Papagayos. Para el enlace troncal planteado a través de un radioenlace  y por fibra óptica, la interconexión con el mismo se plantea con los siguientes esquemas:
\begin{enumerate}
\item[•]En el caso de un enlace troncal por microondas, se interconecta a través de la unidad de radio ODU, utilizando un router de última milla junto a un switch. La Figura \ref{dem-figuras:comobajo_accespoint} ilustra el esquema explicado anteriormente.

\FigureDEM{H}{12}{comobajo_accespoint}{Topología planteada para enlace troncal por radioenlace y red acceso inalámbrica.}


\item[•]Para un enlace troncal a través de fibra óptica se plantea una topología similar la cuál se ilustra en la Figura \ref{dem-figuras:comobajo_fibraoptica} .


\FigureDEM{H}{12}{comobajo_fibraoptica}{Topología planteada para enlace troncal por fibra óptica y red acceso inalámbrica.}


\end{enumerate}


Sobre cada hogar debe colocarse un equipo CPE (\textit{Customer Premises Equipment}) el cuál permite la conexión al servicio desde el AP (Access Point). La estructura general de la red planteada se puede observar en la Figura \ref{dem-figuras:figE7}.

\FigureDEM{H}{11}{figE7}{Estructura General de Red de Acceso Inalámbrica.}

Para determinar el número de APs necesarios, se determina la máxima tasa demandada por un hogar.
Para el cálculo de la tasa necesaria por hogar, se selecciona el pack Oro, el cual posee la tasa de transmisión mayor entre los distintos paquetes. De esta forma, la red de acceso inalámbrica se está sobredimensionando para lograr abastecer un posible crecimiento de demanda de servicios.

Considerando el mayor consumo posible de tasa en un hogar, el cual ocurre con el uso simultaneo de VoIP, Internet y la sintonización de un canal HD y, además, en promedio, existe 1,5 televisores por hogar, se realiza el siguiente análisis.

\begin{itemize}
\item Servicio de televisión:

La tasa de este servicio es totalmente independiente del número de clientes que lo contrata. Al tomar un promedio de 1,5 televisores por hogar, el peor de los casos se presenta cuando se utiliza el canal HD. La tasa de bits calculada se muestra continuación:

\begin{equation}
Tasa=1,5 * 6Mbps= 9Mbps
\end{equation}

\item Internet y VoIP:

La tasa del Plan seleccionado para internet es de 6 Mbps con un índice de simultaneidad de 1:4 y para VoIP de 30 kbps.

La tasa por hogar esta indicada en la Ecuación \ref{tasa_hogares}.
\begin{equation}
Tasa_{HOGAR}=Tasa_{VoIP}+\dfrac{Tasa_{Internet}}{F_{Simultaneidad}} +Tasa_{Canal\hspace{0.15cm}HD}
\label{tasa_hogares}
\end{equation}

\begin{equation}
Tasa_{HOGAR}=0.03 Mbps + \dfrac{6Mbps}{4}+ 1.5*6 Mbps=10.53Mbps
\label{m3}
\end{equation}

A partir de la tasa de televisión, internet y VoIP se requiere que la capacidad total por abonado sea de 10,53 Mbps aproximadamente.
\end{itemize}


Considerando que el ancho de banda demandado en el peor de los casos corresponde a 61 clientes sintonizando un canal. Entonces la tasa máxima consumida se indica en la ecuación \ref{m2}.

\begin{equation}
Tasa_{MAXIMA}=N_{USUARIOS}*Tasa_{HOGAR}=61*10.53Mbps=642,33 Mbps
\label{m2}
\end{equation}

El esquema general de la red corresponde al tipo Punto-Multipunto, en el que un terminal efectúa una transmisión sobre un área en el alcance de múltiples usuarios abonados al servicio.

Al tratarse de una transmisión sobre un medio no guiado, existen algunos aspectos que se deben definir:
\begin{enumerate}
\item[•]El estándar elegido debe permitir algún método de acceso al medio que permita múltiples abonados, tal como TDMA/FDMA/OFDMA, entre otros.
\item[•]La banda en el espectro radioelectrico sobre la cuál se trabaja debe corresponder a ISM (\textit{Industrial, Scientific and Medical}).
\item[•]Debe brindar un marco de seguridad y calidad de servicio (QoS) suficiente para la aplicación. 
\end{enumerate}


Algunas tecnologías inalámbricas en el mercado corresponde a los estándares \textit{WiMax} y \textit{WiFi}, sin embargo en este proyecto se implementa la version 802.11 n del estándar \textit{WiFi} porque cumple con los requerimientos de alcance y ancho de banda necesario (a 50 metros puede brindar una tasa neta de 150 Mbps). Cabe destacar que \textit{WiMAX} se encuentra enfocado más en los operadores del servicio que en los usuarios finales\cite{wifivswimax}.

Teniendo en cuenta la tasa bruta requerida, la tasa real máxima que el estándar puede ofrecer, y que los clientes se encuentran lo suficientemente distribuidos sobre la ciudad. Es posible afirmar que con 5 AP's se logra satifacer tanto la cobertura como la tasa demandada.

\FigureDEM{H}{18}{Captura_lugares}{Distribución de AP's en la Ciudad de Papagayos.}




Teniendo en cuenta el esquema de conexión de las Figuras \ref{dem-figuras:comobajo_accespoint} y \ref{dem-figuras:comobajo_fibraoptica}, se utiliza los siguientes equipos de conectividad.

% Tabla de caracteristicas del router y del switch


% Table generated by Excel2LaTeX from sheet 'Hoja2'
\begin{table}[H]
  \tiny
  \centering
    \begin{tabular}{|c|c|c|c|}
    \hline
    \rowcolor[rgb]{ .773,  .851,  .945} \textbf{Modelo} & \textbf{Fabricante} & \textbf{Nro. Puertos} & \textbf{Costo US\$} \bigstrut\\
    \hline
    ES-10X &   Ubiquiti    & 8 puertos GE / 2 puertos SFP+ & 120\cite{switch} \bigstrut\\
    \hline
    \end{tabular}%
	\caption{Características del switch.}
  \label{tab:caracteristicas-switch}%
\end{table}%



% Table generated by Excel2LaTeX from sheet 'Hoja2'
\begin{table}[H]
  \tiny
  \centering
  
    \begin{tabular}{|c|c|c|c|}
    \hline
    \rowcolor[rgb]{ .773,  .851,  .945} \textbf{Modelo} & \textbf{Fabricante} & \textbf{Nro. Puertos} & \textbf{Costo US\$} \bigstrut\\
    \hline
    EdgeRouter 4 & Ubiquiti & 4 puertos GE / 1 puerto SFP+/ 1 entrada USB para comunicación interna & 120\cite{estefi3} \bigstrut\\
    \hline
    \end{tabular}%
	\caption{Características del router.}
  \label{tab:caracteristicas-router}%
\end{table}%


Para la elección de los dispositivos APs se tuvieron en cuenta 3 candidatos, mostrados en la Tabla \ref{tab:caracteristicas-aps}.

% Table generated by Excel2LaTeX from sheet 'Hoja1'
\begin{table}[H]
  \tiny
  \centering
    \begin{tabular}{|c|c|c|c|c|l|c|c|}
    \hline
    \rowcolor[rgb]{ .773,  .851,  .945} \textbf{Modelo} & \textbf{Fabricante} & \textbf{TX(dBm)} & \textbf{RX(dBm)} & \textbf{Montaje de Equipo} & \multicolumn{1}{c|}{\textbf{Estandar}} &       & \textbf{Costo (US\$)} \bigstrut\\
    \hline
    \multirow{3}[6]{*}{3917i/e Extreme Wireless} & \multirow{3}[6]{*}{Extreme} & \multirow{3}[6]{*}{25} & \multirow{3}[6]{*}{-94} & \multirow{3}[6]{*}{Outdoor} & IEEE 802.11a/b/g & Velocidad de transmision   & \multirow{3}[6]{*}{244} \bigstrut\\
\cline{6-7}          &       &       &       &       & IEEE 802.11n & MIMO  &  \bigstrut\\
\cline{6-7}          &       &       &       &       & IEEE 802.11ac & \textcolor[rgb]{ .125,  .129,  .141}{Garantizar una mayor velocidad de la red} &  \bigstrut\\
    \hline
    \multirow{3}[6]{*}{OSBRiDGBE 5N} & \multirow{3}[6]{*}{4GON solutions} & \multirow{3}[6]{*}{23} & \multirow{3}[6]{*}{-92} & \multirow{3}[6]{*}{Outdoor} & IEEE 802.3 & CSMA / CD & \multirow{3}[6]{*}{150} \bigstrut\\
\cline{6-7}          &       &       &       &       & IEEE 802.3af & Estándar PoE para alimentar equipos &  \bigstrut\\
\cline{6-7}          &       &       &       &       & IEEE 802.11a / n & Velocidad de transmision &  \bigstrut\\
    \hline
    \multirow{3}[6]{*}{ePMP 1000 Force 180} & \multirow{3}[6]{*}{Cambium Networks} & \multirow{3}[6]{*}{30} & \multirow{3}[6]{*}{-90} & \multirow{3}[6]{*}{Outdoor} & IEEE 802.1Q con IEEE 802.1p & VLAN con priorización de tráfico  & \multirow{3}[6]{*}{180 \cite{E14}} \bigstrut\\
\cline{6-7}          &       &       &       &       & IEEE 802.11n & MIMO  &  \bigstrut\\
\cline{6-7}          &       &       &       &       & IEEE 802.11a/b/g & Velocidad de transmision  &  \bigstrut\\
    \hline
    \end{tabular}%
	\caption{Características de los AP's.}
  \label{tab:caracteristicas-aps}%
\end{table}%


%Precio del equipo OSBRIDGE: http://www.sklep.winet.com.pl/osbridgeseria5n/c/24

Estos equipos cumplen con las regulaciones que rigen en el país, ofrecen tasas de transmisión máximas similares, sin embargo se decanta por el último modelo listado (“ePMP 1000 Force 180” de la empresa Cambium Networks), debido a que en la Red de distribución se opta por el uso de dispositivos fabricados por estas empresa, por lo cual es más conveniente para el proceso de capacitación del personal técnico, las posibilidades de trabajo en conjunto con la empresa fabricante incrementan considerablemente, por lo cual es posible obtener beneficios tales como descuentos, o asistencia técnica. Otro motivo importante es la cantidad de información disponible, lo cual adquiere un gran valor en el momento de la resolución de futuros problemas en la red.



El equipo CPE se entrega al abonado en comodato durante el tiempo en el que se asocie al servicio. Considerando que el cuidado del equipo queda bajo responsabilidad del usuario, haciéndose cargo de la reparación de fallas originadas por accidentes tales como: negligencia, abuso, falla eléctrica, causas de fuerza mayor o caso fortuito.

En el análisis de la cobertura del servicio se consideraron dos equipos, los cuales se muestran en la Tabla \ref{tab:caracteristicas-cpe}.


% Table generated by Excel2LaTeX from sheet 'Hoja1'
\begin{table}[H]
  \tiny
  \centering
    \begin{tabular}{|c|c|c|c|c|c|c|c|}
    \hline
    \rowcolor[rgb]{ .773,  .851,  .945} \textbf{Modelo} & \textbf{Fabricante} & \textbf{TX(dBm)} & \textbf{RX(dBm)} & \textbf{Montaje de Equipo} & \textbf{Estandar} &       & \textbf{Costo (US\$)} \bigstrut\\
    \hline
    \multirow{3}[6]{*}{ePMP Force 130} & \multirow{3}[6]{*}{Cambium Networks} & \multirow{3}[6]{*}{28} & \multirow{3}[6]{*}{-88} & \multirow{3}[6]{*}{Outdoor} & IEEE 802.1Q con IEEE 802.1p & VLAN, priorización de tráfico  & \multirow{3}[6]{*}{140 \cite{CPE}} \bigstrut\\
\cline{6-7}          &       &       &       &       & IEEE 802.11n & MIMO  &  \bigstrut\\
\cline{6-7}          &       &       &       &       & IEEE 802.11a/b/g & Velocidad de transmision  &  \bigstrut\\
    \hline
    \multirow{3}[6]{*}{ DLB 5-15} & \multirow{3}[6]{*}{LigoWave} & \multirow{3}[6]{*}{29} & \multirow{3}[6]{*}{-87} & \multirow{3}[6]{*}{Outdoor} & IEEE 802.11 a & Velocidad de transmision de los datos  & \multirow{3}[6]{*}{56} \bigstrut\\
\cline{6-7}          &       &       &       &       & IEEE 802.11ac & \textcolor[rgb]{ .125,  .129,  .141}{Garantizar una mayor velocidad de la red} &  \bigstrut\\
\cline{6-7}          &       &       &       &       & IEEE 802.11n & MIMO  &  \bigstrut\\
    \hline
    \end{tabular}%
	\caption{Características de los CPE's.}
  \label{tab:caracteristicas-cpe}%
\end{table}%


El primer equipo de la Tabla \ref{tab:caracteristicas-cpe} se encuentra elaborado por el fabricante de los AP's, lo cual cabe suponer que haya compatibilidad entre los mismos y haya mayor posibilidad de satisfacer la calidad del servicio, este último aspecto no se encuentra garantizado si se elige el segundo fabricante, de manera que se decide utilizar el equipo ePMP Force 130.

Para la distribución de los tres servicios en el Hogar, debe incorporarse el equipo Modem Gateway Triple Play cuyo modelo es TG789VN V3 de la empresa Technicolor, cuyo precio es de US\$ 50.



\newpage






\begin{thebibliography}{99}

\bibitem{estefi3} Router Ubiquiti: {\tiny \url{https://www.landashop.com/ubn-er-4.html}}

%\bibitem{terreno1duplicado}Terreno Papagayos: \begin{tiny}
%\url{https://terreno.mercadolibre.com.ar/MLA-782002729-lotes-desde-400m2-en-papagayos-_JM#position=1&type=item&tracking_id=902decce-f114-4574-8596-c8c0f75f1595}
%\end{tiny}

\bibitem{switch} Switch Ubiquiti: {\tiny \url{https://articulo.mercadolibre.com.ar/MLA-861437448-ubiquiti-edgeswitch-es-10xp-8-gigabit-poe-out-2-sfp-_JM#position=1&type=item&tracking_id=700f6c13-dbd7-432f-98fe-5439c4324646}}



\bibitem{E14}Access Point:{\tiny \url{ https://articulo.mercadolibre.com.ar/MLA-747386210-epmp-1000-5-ghz-force-180-radio-integrado-_JM?quantity=1# position=2&type=item&tracking_id=286aab7a-b10d-4d0d a5c3-2f5bd57cafd6}}


\bibitem{E2} Apéndice B.

\bibitem{CPE}CPE Cambium ePMP Force 130: \begin{tiny}
\url{https://articulo.mercadolibre.com.ar/MLA-815770606-cambium-epmp-force-130-5-ghz-sm-c050900c512a-_JM?matt_tool=68536632&matt_word=&matt_source=google&matt_campaign_id=10375307423&matt_ad_group_id=102828799643&matt_match_type=b&matt_network=g&matt_device=c&matt_creative=444135776619&matt_keyword=&matt_ad_position=&matt_ad_type=&matt_merchant_id=&matt_product_id=&matt_product_partition_id=&matt_target_id=aud-753845548730:dsa-19959388920&gclid=CjwKCAiA7939BRBMEiwA-hX5J36c9bxCCu-Ky1MUFAn4c4U871GQkqQ1fNjSv8M6TCop5bOSgqT49RoC5aYQAvD_BwE}
\end{tiny}


\bibitem{wifivswimax} Comparación entre estandares \textit{WiFi} y \textit{WiMAX}. \begin{tiny}
\url{http://www.ptolomeo.unam.mx:8080/xmlui/bitstream/handle/132.248.52.100/164/A8.pdf?sequence=8}
\end{tiny}

\end{thebibliography}

\newpage