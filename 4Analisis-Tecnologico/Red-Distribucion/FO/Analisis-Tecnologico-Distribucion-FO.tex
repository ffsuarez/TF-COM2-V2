\section{Red de Distribución por Fibra Óptica}


Para otorgar los servicios del tipo \textit{Triple Play} a los abonados de la localidad de Papagayos, desde la localidad de Carpintería, se plantea un enlace troncal a través de un tendido de fibra óptica.

El objetivo de esta red es la vinculación de la oficina central (ya instalada), en la localidad de Carpintería, con los equipos receptores, ubicados en Papagayos, con el fin de  proveer dichos servicios.

%Para realizar la transferencia entre el acceso y los servicios del tipo \textit{Triple Play} a los abonados de la localidad de Papagayos se plantea un enlace a través de un tendido de fibra óptica.

Para la implementación de la red troncal es necesario la elección del tipo, el estándar y la clase de tendido del cable óptico.


La elección del tipo de fibra óptica depende, en primer lugar, de la distancia del enlace.


La utilización de la fibra multimodo es adecuada para distancias de enlaces cortas, generalmente hasta 2 Km.
En cambio, la fibra monomodo es utilizada para implementaciones de redes de larga distancia y es apta para velocidades de transmisión mayores a la fibra multimodo.


La Figura \ref{dem-figuras:fig2000} muestra un mapa de las localidades de Carpintería y Papagayos. La distancia en línea recta  entre ambas localidades ronda alrededor de 30 Km, lo cual infiere la necesidad de la utilización de una fibra monomodo para el establecimiento del enlace.

\FigureDEM{H}{8}{fig2000}{Mapa Papagayos - Carpintería}


% ########################################################################
% ########################################################################




Entre las distintas recomendaciones ofrecidas por la UIT para una fibra óptica monomodo, se opta por la elección de fibras que cumplen con el estándar G.652. Dicha recomendación permite operar en la región de 1550 nm y presenta un gran éxito comercial.


Para establecer el enlace troncal se opta por utilizar una fibra optica del tipo monomodo. Entre las diferentes fibras ópticas monomodo establecidas por el UIT-T se incluyen las recomendaciones: G.652, G.653, G.654, G.655, G.656 y la G.657. Una descripción resumida se encuentra en la Tabla \ref{tab:estandares}: 

%De acuerdo a las bandas de operación especificadas y la tecnología CWDM, se opta por la utilización del estándar UIT-T G.652 D, el cual elimina el pico de agua para una operación de espectro completo. Dicho estándar puede utilizarse en las regiones de longitud de onda de 1310 nm y 1550 nm y admite la transmisión por multiplexación por división de longitud de onda gruesa (CWDM).
%
%
%https://beyondtech.us/blogs/beyondtech-en-espanol/cuales-son-las-diferencias-entre-las-fibras-opticas-monomodo-g-652-y-g-655
%
%https://community.fs.com/es/blog/is-g652-single-mode-fiber-your-right-choice.html

 




% Table generated by Excel2LaTeX from sheet 'Hoja1'
\begin{table}[H]
  \tiny
  \centering
    \begin{tabular}{|c|p{6.78em}|p{22.945em}|c|}
    \hline
    \rowcolor[HTML]{C5D9F1} \multicolumn{1}{|p{9.055em}|}{\textbf{Recomendación}} & \textbf{Tipos} & \textbf{Atenuación} & \multicolumn{1}{p{41.335em}|}{\textbf{Uso}} \bigstrut\\
    \hline
    \multicolumn{1}{|c|}{\multirow{4}[8]{*}{UIT- T G.652}} & UIT- T G.652.A & Menos de 0.5 sobre 1310/1550nm & \multicolumn{1}{c|}{\multirow{4}[8]{*}{LAN, MAN, redes de acceso y transmisión CWDM.}} \bigstrut\\
\cline{2-3}          & UIT- T G.652.B & Menos de 0.4 sobre 1310/1550nm &  \bigstrut\\
\cline{2-3}          & UIT- T G.652.C & \multirow{2}[4]{*}{Menos de 0.4 de 1310 a 1625nm} &  \bigstrut\\
\cline{2-2}          & UIT- T G.652.D & \multicolumn{1}{c|}{} &  \bigstrut\\
    \hline
    \multicolumn{2}{|p{17.835em}|}{UIT- T G.653} & Menos de 0.35/0.55 sobre 1550/1310nm & \multicolumn{1}{p{41.335em}|}{Sistemas de transmisión que utilizan amplificadores de fibra dopados con erbio.} \bigstrut\\
    \hline
    \multicolumn{2}{|p{17.835em}|}{UIT- T G.654} & Menos de 0.25 sobre 1550nm & \multicolumn{1}{p{41.335em}|}{Sistemas submarinos utilizando fibra óptica con corte desplazado.} \bigstrut\\
    \hline
    \multicolumn{2}{|p{17.835em}|}{UIT- T G.655} & Menos de 0.27/0.35 sobre 1550/1625nm & \multicolumn{1}{p{41.335em}|}{Sistemas submarinos utilizando fibra óptica de dispersión desplazada no nula.} \bigstrut\\
    \hline
    \multicolumn{2}{|p{17.835em}|}{UIT- T G.656} & Menos de 0.4/0.35/0.5 sobre 1460/1550/1625nm & \multicolumn{1}{p{41.335em}|}{Sistemas de larga distancia que utilizan rangos de longitud de onda desde 1460 a 1625 nm.\newline{}} \bigstrut\\
    \hline
    \multicolumn{1}{|c|}{\multirow{2}[4]{*}{UIT- T G.657}} & Clase A & Menos de 0.5/0.3 sobre 1310/1550nm & \multicolumn{1}{c|}{\multirow{2}[4]{*}{Fibra óptica insensible a la pérdida \newline{} ocasionada por curvaturas, diseñada para Redes de Acceso.}} \bigstrut\\
\cline{2-3}          & Clase B & Menos de 0.5/0.3/0.4 sobre 1310/1550/1625nm &  \bigstrut\\
    \hline
    \end{tabular}%
    \caption{Estándares aplicados en las fibras opticas monomodo.}
  \label{tab:estandares}%
\end{table}%



Se opta por la utilización del estándar UIT-T G.652 D, la cual, además de ser la más popular en el mercado, ofrece la posibilidad de futuras expansiones de la red al eliminar el pico de agua para una operación de espectro completo y permitiendo la implementación de WDM. Esta implementación utiliza las regiones de longitud de onda de 1310 nm y 1550 nm.



% ########################################################################
% ########################################################################

 

Una vez presentado el estándar de la fibra óptica, se analizan tres modelos de fibras ópticas ofrecidas en el mercado, las mismas se presentan en la Tabla \ref{tab:fo-analizadas}:



% Table generated by Excel2LaTeX from sheet 'Hoja3'
\begin{table}[H]
  \tiny  
  \centering
    \begin{tabular}{|c|c|c|c|c|cc|}
    \hline
    \rowcolor[rgb]{ .773,  .851,  .945} \textbf{Modelo} & \textbf{Fabricante} & \textbf{Nro. de fibras/tubo} & \textbf{Atenuación Típica en dB/km} & \textbf{Costo en US\$} & \multicolumn{2}{c|}{\textbf{Aspectos Constructivos}} \bigstrut\\
    \hline
    \multirow{6}[2]{*}{\textbf{SM Mini-Spam 424}} & \multirow{6}[2]{*}{AFL} & \multirow{6}[2]{*}{6} & \multirow{6}[2]{*}{0.35/0.25 (1310nm/1550nm)} & \multirow{6}[2]{*}{2,95} & \multicolumn{2}{c|}{Cable tipo ADSS} \bigstrut[t]\\
          &       &       &       &       & \multicolumn{2}{c|}{} \\
          &       &       &       &       & \multicolumn{2}{c|}{} \\
          &       &       &       &       & \multicolumn{2}{c|}{} \\
          &       &       &       &       & \multicolumn{2}{c|}{} \\
          &       &       &       &       & \multicolumn{2}{c|}{} \bigstrut[b]\\
    \hline
    \multirow{6}[2]{*}{\textbf{ADSS-5c86}} & \multirow{6}[2]{*}{GT Optical} & \multirow{6}[2]{*}{>= 10} & \multirow{6}[2]{*}{0.38/0.22 (1310nm/1550nm)} & \multirow{6}[2]{*}{0,65} & \multicolumn{2}{c|}{Cable tipo ADSS} \bigstrut[t]\\
          &       &       &       &       & \multicolumn{2}{c|}{Revestimiento PE-LSZH } \\
          &       &       &       &       & \multicolumn{2}{c|}{(Polietileno, Retardante de llama} \\
          &       &       &       &       & \multicolumn{2}{c|}{Materiales no halógenos)} \\
          &       &       &       &       & \multicolumn{2}{c|}{Hilos de kevlar de aramida} \\
          &       &       &       &       & \multicolumn{2}{c|}{} \bigstrut[b]\\
    \hline
    \multirow{9}[2]{*}{\textbf{M9W510T}} & \multirow{9}[2]{*}{Belden} & \multirow{9}[2]{*}{6} & \multirow{9}[2]{*}{Máx: 0.4/0.3(1310nm/1550nm)} & \multirow{9}[2]{*}{1,7} & \multicolumn{2}{c|}{Cable tipo ADSS (Loose Tube)} \bigstrut[t]\\
          &       &       &       &       & \multicolumn{2}{c|}{Revestimiento PBT (Termoplastico)} \\
          &       &       &       &       & \multicolumn{2}{c|}{Hilos de kevlar de aramida} \\
          &       &       &       &       & \multicolumn{2}{c|}{} \\
          &       &       &       &       & \multicolumn{2}{c|}{} \\
          &       &       &       &       & \multicolumn{2}{c|}{} \\
          &       &       &       &       & \multicolumn{2}{c|}{} \\
          &       &       &       &       & \multicolumn{2}{c|}{} \\
          &       &       &       &       & \multicolumn{2}{c|}{} \bigstrut[b]\\
    \hline
    \end{tabular}%
  \caption{Fibras Ópticas Analizadas.}
  \label{tab:fo-analizadas}%
\end{table}%


Considerando el 



\begin{itemize} 


\item[•] El recorrido comienza desde la Oficina Central ubicada en Carpintería. Su ubicación se encuentra detallada en la Figura \ref{dem-figuras:oficina_central} . 

\FigureDEM{H}{12}{oficina_central}{Trayectoria del enlace troncal.}




\item[•] La fibra se despliega a lo largo de la Ruta Provincial N $^{\circ}$1, como se observa en la Figura \ref{dem-figuras:fig2004}, el trayecto tiene una extensión de 30,2 Km aproximadamente \cite{secretaria}.

\FigureDEM{H}{10}{fig2004}{Trayectoria del enlace troncal.}

\item[•] La terminación de la misma se establece en la intersección entre la Ruta Provincial N $^{\circ}$1 y la Avenida San Pedro, en la localidad de Papagayos, tal y como se muestra sobre la Figura \ref{dem-figuras:terminacion_fibra} .

\FigureDEM{H}{16}{terminacion_fibra}{Terminación del enlace troncal.}
\end{itemize}

 


A través de la herramienta que ofrece la Secretaría de Energía de la Nación se observa un tendido eléctrico al costado de la Ruta Provincial Nº1, el cual se extiende hasta Papagayos, atravesando Villa Larca, Cortaderas. Esto facilita la colocación del tendido de fibra óptica ya que se utilizarán los postes de la linea de media tensión ya instalados.
Como el tendido se ubica en las cercanías de la ruta, ofrece facilidad en cuanto al acceso del mismo frente a cualquier problemática del servicio (por ejemplo, corte de fibra).



En dicha instalación, además de los 30,2 Km aproximados de fibra óptica, es necesario adicionar cierto extra de fibra debido a la catenaria, el margen de seguridad y la geometría del tendido eléctrico.

Una primera aproximación posible para longitud total de la fibra puede calcularse añadiendo un 10\% de la longitud total del mismo, considerando los extras mencionados anteriormente. Así, son necesarios 33,22 Km de cable óptico en total.






\newpage

%-------------------------------------------------------------
\begin{thebibliography}{99}
\bibitem{secretaria}Dirección Nacional de Información Energética. Secretaria de Energía: \url{https://sig.se.gob.ar/visor/visorsig.php?t=1}

\end{thebibliography}
%-------------------------------------------------------------


\newpage