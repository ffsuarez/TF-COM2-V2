\section{Red de Distribución por Radioenlace}


Para brindar el servicio de \textit{Triple Play} a los abonados de la localidad de Papagayos se plantea un radioenlace punto a punto de Mediana Capacidad (cuya frecuencia se determina entre 3 Ghz y 15 Ghz), digital, de largo alcance y alto ancho de banda, entre dos estaciones terminales ubicadas en las dos localidades mencionadas. Sin embargo, como se puede apreciar en la Figura \ref{dem-figuras:fig13} no hay linea de vista entre las mismas, por lo cual se utiliza una estación repetidora.

\FigureDEM{H}{12}{fig13}{Perfil topográfico entre estaciones terminales.}

Se considera uilizar la torre previamente instalada en la oficina central ubicada en Carpintería. De esta forma se plantea la ubicación y equipamiento del repetidor y receptor.


Para la viabilidad del radioenlace se establece que debe haber un despeje de la zona de Fresnel de al menos el 60 \% y un margen de recepción de 20 dB. Dicho cálculo se realiza utilizando software desarrollado para esta tarea.


\subsection{Ubicación de estaciones}

\begin{enumerate}
\item[•]La estación transmisora se encuentra establecida sobre la ciudad de Carpintería, la Figura \ref{dem-figuras:estacion_carpinteria} permite identificar la posición donde esta ubicada la instalación de la misma.
\FigureDEM{H}{12}{estacion_carpinteria}{Ubicación de estación transmisora.}


\item[•]La estación repetidora se coloca aproximadamente a 8 km al este de Concarán, sobre la ruta provincial N$^{\circ}$ 6 . La Figura \ref{dem-figuras:estacion_repetidora} ilustra la ubicación de la misma.
Dicho lugar es el de mayor cercanía entre las dos localidades que permite la viabilidad del enlace, posee acceso a energía eléctrica, y se encuentra en una posición estratégica para futuras ampliaciones sobre ciudades vecinas.

\FigureDEM{H}{12}{estacion_repetidora}{Ubicación de estación repetidora.}


\item[•]La estación receptora se coloca a la altura de la curva que atraviesa la ciudad de Papagayos. La Figura \ref{dem-figuras:estacion_papagayosPNG} ilustra la posición de la misma.

\FigureDEM{H}{12}{estacion_papagayosPNG}{Ubicación de estación receptora.}

\end{enumerate}

\subsection{Equipos necesarios}

Se precisan torres de telecomunicación que permitan el soporte de la antena a las alturas especificadas en la Tabla \ref{tab:alturas-torres}. Se decide colocar una torre autosoportada, ya que las mismas adquieren una cimentación adecuada para terrenos ubicados en áreas urbanas y cerros, de forma que logren resistir las fuerzas a las que están sometidas.

\begin{table}[H]
    \begin{tabular}{|l|l|}
    \hline
    \rowcolor[rgb]{ .773,  .851,  .945} \textbf{Carpinteria} & \cellcolor[rgb]{ 1,  1,  1} m \bigstrut\\
    \hline
    \rowcolor[rgb]{ .773,  .851,  .945} \textbf{Repetidor} & \cellcolor[rgb]{ 1,  1,  1} m \bigstrut\\
    \hline
    \rowcolor[rgb]{ .773,  .851,  .945} \textbf{Papagayos} & \cellcolor[rgb]{ 1,  1,  1} m \bigstrut\\
    \hline
    \end{tabular}%
  \centering
  \caption{Alturas de las torres.}
  \label{tab:alturas-torres}%
\end{table}%

La banda de frecuencias a las cuales se establece el enlace corresponde a 7725 Mhz hasta 8500 Mhz. A pesar que la mayoría de los equipos de transmisión y recepción analizados ofrecen una antena incorporada, no adquieren la ganancia necesaria en la banda especificada. Se decide utilizar una antena exterior, cuyas características se ilustran en la Tabla \ref{tab:caracteristicas-antena} .

\begin{table}[H]
\begin{center}
\begin{tabular}{|c|l|l|l|}
\hline
\rowcolor[HTML]{C5D9F1}{ \textbf{Diametro{[}m{]}}} & { \textbf{Ganancia{[}dBi{]}}} & { \textbf{\footnotesize{Discriminación de Polarización Cruzada}{[}dB{]}}} & { \textbf{VSWR(R.L.,[dB])}} \\ \hline
1,2                                             & \multicolumn{1}{c|}{37,9}                         & \multicolumn{1}{c|}{32}                                                        & \multicolumn{1}{c|}{1,15(23,1)}               \\ \hline
\end{tabular}
\caption{Características principales Antena.}
\label{tab:caracteristicas-antena}
\end{center}
\end{table}

 
Los equipos de transmisión y recepción pueden adquirir diversas configuraciones:

\begin{enumerate}
\item[•]All Outdoor: Tanto la unidad de radio, como el módem que interconecta la radio con el backbone, se colocan sobre la torre.

\item[•]All Indoor: Tanto la unidad de radio, como el módem que interconecta la radio con el backbone, se colocan sobre la caseta.

\item[•]Split Mount: La unidad de radio se coloca sobre la torre, mientras que el módem se coloca en la caseta.
\end{enumerate}


Se ha analizado varios equipos del mercado y sobre la Tabla \ref{tab:caracteristicas-eq} se ilustran las características principales que se tomaron en cuenta tanto para la verificación de la viabilidad del enlace, como asi también las diversas tecnologías que ofrecen.

% Table generated by Excel2LaTeX from sheet 'Hoja1'
\begin{table}[H]
  \tiny
  \centering  
    \begin{tabular}{|c|c|c|c|c|c|p{10.945em}|c|}
    \hline
    \rowcolor[rgb]{ .773,  .851,  .945} \textbf{Equipo} & \textbf{Fabricante} & \textbf{TX (dBm)} & \textbf{RX (dBm)} & \textbf{Montaje de Equipo} & \multicolumn{2}{c|}{\textbf{Estandar }} & \textbf{Costo (US\$)} \bigstrut\\
    \hline
    \multirow{6}[2]{*}{\textbf{PTP 820-S}} & \multirow{6}[2]{*}{Cambium Networks} & \multirow{6}[2]{*}{23} & \multirow{6}[2]{*}{-66} & \multirow{6}[2]{*}{All-outdoor} & IEEE 802.3 & Posibilidad de & \multirow{6}[2]{*}{3500} \bigstrut[t]\\
          &       &       &       &       & IEEE 802.3ac & \multicolumn{1}{c|}{implementar VLAN's} &  \\
          &       &       &       &       & IEEE 802.1Q & \multicolumn{1}{c|}{Multicasting, Prioridad} &  \\
          &       &       &       &       & IEEE 802.1p & \multicolumn{1}{c|}{de tráfico} &  \\
          &       &       &       &       & IEEE 802.1ad & \multicolumn{1}{r|}{} &  \\
          &       &       &       &       & IEEE 802.3ad & \multicolumn{1}{r|}{} &  \bigstrut[b]\\
    \hline
    \multirow{6}[2]{*}{\textbf{FibeAir IP-20C}} & \multirow{6}[2]{*}{Ceragon} & \multirow{6}[2]{*}{24} & \multirow{6}[2]{*}{-75} & \multicolumn{1}{c|}{\multirow{6}[2]{*}{All-Outdoor o Split Mount}} & IEEE 802.3 & Posibilidad de & \multirow{6}[2]{*}{4000} \bigstrut[t]\\
          &       &       &       &       & IEEE 802.3ac & \multicolumn{1}{c|}{implementar VLAN's} &  \\
          &       &       &       &       & IEEE 802.1Q & \multicolumn{1}{c|}{Multicasting, Prioridad} &  \\
          &       &       &       &       & IEEE 802.1p & \multicolumn{1}{c|}{de tráfico} &  \\
          &       &       &       &       & IEEE 802.1ad & \multicolumn{1}{r|}{} &  \\
          &       &       &       &       & IEEE 802.3ad & \multicolumn{1}{r|}{} &  \bigstrut[b]\\
    \hline
    \multirow{9}[2]{*}{\textbf{iPASOLINK 400}} & \multirow{9}[2]{*}{NEC} & \multirow{9}[2]{*}{24} & \multirow{9}[2]{*}{-66} & \multirow{9}[2]{*}{Split-Mount} & IEEE802.3z & Además de las  & \multirow{9}[2]{*}{4500} \bigstrut[t]\\
          &       &       &       &       & IEEE802.1ag & características  &  \\
          &       &       &       &       & IEEE802.3ad & anteriores, se suma la &  \\
          &       &       &       &       & IEEE802.3ab &  posibilidad de  &  \\
          &       &       &       &       & IEEE802.1Q & tratamiento de  &  \\
          &       &       &       &       & IEEE802.1ad & múltiples enlaces  &  \\
          &       &       &       &       & IEEE802.1w & punto a punto como si  &  \\
          &       &       &       &       & IEEE802.1AX & fuera uno solo  &  \\
          &       &       &       &       & IEEE802.3ah & (802.1AX) &  \bigstrut[b]\\
    \hline
    \end{tabular}%
    \caption{Características equipos transmisión y recepción de radioenlace.}
  \label{tab:caracteristicas-eq}%
\end{table}%








Tras analizar los equipos se pondera el precio como un factor crucial para la decisión, también se tiene en cuenta la disponibilidad de asistencia técnica e información ofrecida. Bajo estas consideraciones se decide utilizar el equipo Cambium PTP 820-S, este es el modelo de menor coste, con locales ubicados en el país y luego de consultar el manual del modelo se aprecia una gran cantidad de información sobre la instalación, configuraciones posibles y alternativas para ofrecer un servicio de mejor calidad.

\newpage