\section{Red de Distribución por Radioenlace}


Para brindar el servicio de \textit{Triple Play} a los abonados de la localidad de Papagayos se plantea un radioenlace punto a punto de Mediana Capacidad (cuya frecuencia se determina entre 3 Ghz y 15 Ghz), digital, de largo alcance y alto ancho de banda, entre dos estaciones terminales ubicadas en las dos localidades mencionadas. Sin embargo, como se puede apreciar en la Figura \ref{dem-figuras:fig13} no hay linea de vista entre las mismas, por lo cual se utiliza una estación repetidora.


Se realiza la suposición que se utiliza la torre previamente instalada en la oficina central ubicada en Carpintería. De esta forma se plantea la ubicación y equipamiento del repetidor y receptor.


\FigureDEM{H}{12}{fig13}{Perfil topográfico entre estaciones terminales.}
%Las ubicaciones detalladas de las estaciones se presentan en las Figuras \ref{dem-figuras:estacion_carpinteria}, \ref{dem-figuras:estacion_repetidora} y \ref{dem-figuras:estacion_papagayosPNG} .
\begin{enumerate}
\item[•]La estación transmisora se encuentra establecida sobre la ciudad de Carpintería, la Figura \ref{dem-figuras:estacion_carpinteria} permite identificar la posición donde esta ubicada la instalación de la misma.
\FigureDEM{H}{12}{estacion_carpinteria}{Ubicación de estación transmisora.}


\item[•]La estación repetidora se coloca sobre las cercanías de una escuela ubicada sobre la ruta provincial N $^{\circ}$ 6, aproximadamente a 8 km al este de Concarán. La Figura \ref{dem-figuras:estacion_repetidora} ilustra la ubicación de la misma.
\FigureDEM{H}{12}{estacion_repetidora}{Ubicación de estación repetidora.}


\item[•]La estación receptora se coloca a la altura de la curva que atraviesa la ciudad de Papagayos. La Figura \ref{dem-figuras:estacion_papagayosPNG} ilustra la posición de la misma.

\FigureDEM{H}{12}{estacion_papagayosPNG}{Ubicación de estación receptora.}
\end{enumerate}

