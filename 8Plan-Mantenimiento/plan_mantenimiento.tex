

\subsection{Equipos, Herramientas y Recursos utilizados}

El plan de mantenimiento es llevado a cabo por un servicio  tercerizado, razon por la cual no se precisa especificar los equipos utilizados para efectuar dichas actividades . 

%Preguntar al profe




%Componentes del conjunto de retención para el cableado ADSS: 
%Para el conjunto de retención del cableado de fibra óptica ADSS se prevée que se utilizen los siguientes componentes:

%\begin{itemize}
%\item[•]Preformado de protección (camisa).
%\item[•]Preformado de retención.
%\item[•] Elementos de Retención:
%
%\begin{itemize}
%\item[•]Ménsula de Retención/ Suspensión.
%\item[•]Brazo de Extensión.
%\item[•]Grillete Tipo Guarda Cabo.
%\item[•]Componentes del conjunto de suspensión.
%\item[•]Preformado de protección (camisa)
%\item[•]Preformado de suspensión.
%\item[•]Elementos de Retención
%\begin{itemize}
%\item[i]Ménsula de Suspensión
%\item[ii]Guarda cabo tipo aro
%\end{itemize}
%
%\end{itemize}
%
%\item[•]El sistema de fijación de cables.
%\item[•]Para la suspensión de la fibra óptica a través de morsetos.
%
%\end{itemize}





%-------------------------------------------------
\subsection{Equipo de Trabajo}


Se estima la necesidad de 1 técnico capacitado para la utilización de las herramientas provistas para la red y de planta externa en general. Deben conocer los protocolos de mantenimiento predictivo y correctivo de la infraestructura de la red. 


Además se precisa la presencia de una persona que esté debidamente capacitada para las siguientes tareas:

\begin{itemize}

\item[•]Manejo del software de gestión y monitoreo de la red.

\item[•]Configuración de enlaces por fibra óptica en función a las interfaces provistas con los equipos. 

\item[•]Provisionamiento de nuevos abonados. 

\item[•]Asignación de perfiles de tráfico a los abonados. 

\item[•]Copias de respaldo de las configuraciones de la electrónica de red. 


\item[•]Conocimientos sólidos sobre IP versión 6 y versión 4.


\item[•]Configuración de seguridad IP.

\item[•]Análisis de fallas en la red.


\end{itemize}





%-------------------------------------------------



\subsection{Actividades de Mantenimiento}
Al considerar que las actividades de instalación ya se encuentran realizadas, la Figura \ref{dem-figuras:mantenimiento} se presenta un esquema de las tareas correspondientes al mantenimiento preventivo de la red.

\FigureDEM{H}{15}{mantenimiento}{Actividades de mantenimiento preventivo.}

%\begin{enumerate}
%\item[1]Inspeccionar la limpieza de la fibra con equipos especializados como microscopios o sondas de inspección de video.
%
%Realizar la comprobación de perdida en la fibra con un set de pruebas de perdidas ópticas (OLTS, del ingles Optical Loss Test Sets).
%
%
%\item[2] Realizar la comprobación de los niveles de pérdida de inserción de cables de interconexión.
%
%\item[3]  Realizar las mediciones de:
%
%\begin{enumerate}
%\item[•]Pérdidas de potencia óptica en el enlace
%\item[•]Perdida de inserción en conectores.
%\item[•]Reflectancia en adaptadores y conectores.
%\item[•]Perdida de retorno o potencia reflejada en conectores y adaptadores. (ORL)
%\item[•]Atenuación (perdidas por kilómetro).
%\item[•]Longitud del enlace.
%\item[•]Polaridad.
%\item[•]Macro y microcurvaturas.
%\end{enumerate}
%
%
%
%\end{enumerate}
%
%
%\begin{enumerate}
%\item[4]Asegurar que las fibras y los patch cords que entran y salen de los ODF's no sobrepasen el radio de curvatura especificado.
%
%\item[5]Verificar que los puertos no utilizados en el ODF cuentan con los tapones contra acumulación de polvo.
%
%
%\item[6]Se debe revisar los márgenes de pérdidas ópticas que aceptan los dispositivos adquiridos en cada punto y confrontarlos con los valores obtenidos en las mediciones.
%
%
%\item[7]  Se debe verificar la estabilidad de los postes, observando que los cables de soporte se encuentren tensionados, así como los extremos de terminales estén instalados.
%
%\item[8] Verificar el estado del cable de fibra óptica entre cada una de las torres, torrecillas, postes y subestaciones. 
%
%
%
%
%
%\end{enumerate}
%
%
%
%
%\begin{enumerate}
%
%\item[9]Limpieza total de los racks de comunicaciones.
%
%
%
%\item[10]Inspeccionar detenidamente que las conexiones, conectores y patch cords, de los Switch de fibra óptica se encuentren en buen estado y conectado como es debido, sin sobrepasar los radios de curvatura, o con conectores sueltos.
%
%
%
%\item[11]Limpiar los conectores SC y LC que se insertan a los porta modulos Mini GBIC de los Switch de fibra, así como las fibras que ingresan al ODF y sus puertos.
%
%
%
%
%
%
%
%\item[12]  El mismo procedimiento de limpieza y revisión que se realiza en los racks de telecomunicaciones debe ser realizado en las cajas de empalme de las torres,  torrecillas, postes y subestaciones.
%
%
%
%
%
%
%\end{enumerate}





\subsubsection{Procedimiento de Mantenimiento Correctivo}
%Cuando la red se encuentra en condiciones normales de funcionamiento, se recopila de manera continua o
%periódica la información sobre la calidad de funcionamiento. Estos datos pueden utilizarse para detectar condiciones agudas de avería.
%
%El cliente realizara la solicitud del servicio correctivo a través de las vías de comunicación que se hayan establecido. Se tomará como hora de solicitud del servicio, la hora en que el proveedor de servicio se considere fehacientemente notificado, en cuyo caso se asignará al caso un determinado número de servicio. 
%
%A tal efecto se designará al menos un funcionario técnico para que sean el responsable de gestionar este tipo de eventos.
%
%Los tiempos para controlar la respuesta del servicio técnico se tomaran a partir que le sea asignado un número de caso al servicio solicitado, cualquiera sea la vía de comunicaciones que se use.

El mantenimiento correctivo se establece a partír de algún evento que puede surgir tras un 
reclamo de un cliente, o a partir de alguna anomalía durante el proceso de mantenimiento 
preventivo.

De acuerdo a la naturaleza de la falla se dispondrán los recursos y el personal técnico 
necesario. Si es una falla de hardware el personal se dirigirá al lugar físico, mientras
que si es de software el personal puede operar de forma remota.


El tiempo máximo establecido para la puesta en normal funcionamiento de los sitios alcanzados por este servicio, será de 72 horas corridas a partir del reporte del evento. Se toma como finalización del evento, la puesta en régimen operativo del sitio en cuestión, con los desperfectos solucionados. 

Este contrato de servicio de mantenimiento no incluye:

\begin{itemize}

\item[•]La sustitución de cualquier material de consumo de los equipos periféricos propiedad del Usuario.

\item[•]La reparación de equipos conectados directa o indirectamente a los equipos.

\item[•]La reparación de fallas que se hayan derivado como resultado de accidentes tales como: negligencia, abuso, falla de energía eléctrica, falla de aire acondicionado, humedad o procedimientos inadecuados en la operación y manejo del equipos, así como problemas ocasionados por los medios de transmisión públicos o privados, y los ocasionados por causas de fuerza mayor o caso fortuito.




\end{itemize}


El equipo técnico deberá informar fehacientemente mediante un Formato de Orden de Trabajo ilustrada en la Tabla \ref{tab:ordentrabajo} el cual contiene la información relevante sobre la ejecución de los trabajos realizados, sectores involucrados y toda otra información que tenga que ver con los servicios y trabajos objeto de la contratación.



% Table generated by Excel2LaTeX from sheet 'Hoja1'
\begin{table}[H]
  \centering
  \scriptsize
    \begin{tabular}{|c|c|c|c|c|c|}
    \hline
    \multicolumn{6}{|c|}{\cellcolor[HTML]{C5D9F1}\textbf{Formato de Orden de Trabajo}} \bigstrut\\
    \hline
    \multicolumn{3}{|c|}{\textbf{Ubicación de Sitio de Trabajo}} & \multicolumn{3}{c|}{\textbf{Fecha y Tiempo de Trabajo}} \bigstrut\\
    \hline
    \multicolumn{1}{|l|}{\textbf{Dirección}} & \multicolumn{1}{l|}{\textbf{Localidad}} & \multicolumn{1}{l|}{\textbf{Teléfono}} & \multicolumn{1}{l|}{\textbf{Fecha}} & \multicolumn{1}{l|}{\textbf{Hora Inicio}} & \multicolumn{1}{l|}{\textbf{Hora Finalización}} \bigstrut\\
    \hline
    \multirow{2}[2]{*}{} & \multirow{2}[2]{*}{} & \multirow{2}[2]{*}{} & \multirow{2}[2]{*}{} & \multirow{2}[2]{*}{} & \multirow{2}[2]{*}{} \bigstrut[t]\\
          &       &       &       &       &  \bigstrut[b]\\
    \hline
    \multicolumn{3}{|c|}{\textbf{Datos de Usuario}} & \multicolumn{3}{c|}{\textbf{Número de Orden}} \bigstrut\\
    \hline
    \multicolumn{2}{|c|}{\textbf{Nombre}} & \multicolumn{1}{l|}{\textbf{Nº abonado}} & \multicolumn{3}{c|}{\multirow{2}[4]{*}{}} \bigstrut\\
\cline{1-3}    \multicolumn{2}{|c|}{} &       & \multicolumn{3}{c|}{} \bigstrut\\
    \hline
    \multicolumn{3}{|c|}{\textbf{Actividad a realizar}} & \multicolumn{3}{c|}{\textbf{Equipos a Utilizar}} \bigstrut\\
    \hline
    \multicolumn{3}{|c|}{\multirow{2}[2]{*}{}} & \multicolumn{3}{c|}{\multirow{2}[2]{*}{}} \bigstrut[t]\\
    \multicolumn{3}{|c|}{} & \multicolumn{3}{c|}{} \bigstrut[b]\\
    \hline
    \multicolumn{6}{|c|}{\textbf{Observaciones}} \bigstrut\\
    \hline
    \multicolumn{6}{|c|}{\multirow{2}[2]{*}{}} \bigstrut[t]\\
    \multicolumn{6}{|c|}{} \bigstrut[b]\\
    \hline
    \multicolumn{3}{|c|}{\textbf{Firma Supervisor}} & \multicolumn{3}{c|}{\textbf{Firma Trabajador/es}} \bigstrut\\
    \hline
    \multicolumn{3}{|c|}{\multirow{2}[2]{*}{}} & \multicolumn{3}{c|}{\multirow{2}[2]{*}{}} \bigstrut[t]\\
    \multicolumn{3}{|c|}{} & \multicolumn{3}{c|}{} \bigstrut[b]\\
    \hline
    \end{tabular}%
    \caption{Formato Orden de Trabajo.}
  \label{tab:ordentrabajo}%
\end{table}%







\subsection{Costo de Plan de Mantenimiento}

El cálculo para el costo del plan de mantenimiento tiene en cuenta el conjunto de tareas preventivas de mantenimiento mensual de la instalación con el fin de aumentar al máximo posible la vida útil de la instalación. En este cálculo de costo también se tiene en cuenta las intervenciones debido a roturas en alguna parte de la red. 

\subsubsection{Mantenimiento Preventivo}

El mantenimiento mensual programado posee un tiempo de duración de 1 días trabajando una jornada de 8 horas. El trabajo se realiza por 1 técnicos y teniendo en cuenta el precio de hora de trabajo de un técnico \cite{costo2tec}, el costo de mantenimiento será de:

1 técnicos: 8 horas 
\begin{equation}
Costo= US\$5* 8 horas= US\$40
\end{equation}

El mantenimiento involucra viajes de una localidad a la otra lo cual se estima 1 día recorre 66 km  entre viajes de ida y vuelta. El vehículo utilizado será alquilado durante un tiempo determinado lo cual permite reducir costos y destinar presupuesto en otras secciones de la empresa. El costo de la camioneta por día US \$ 33 \cite{costo20}. La camioneta tiene un consume de 7,3 litros cada 100km. 

Teniendo en cuenta los datos anteriores:

%\begin{equation}
%
%5*33 = 165
%
%\end{equation}

\begin{enumerate}


\item El alquiler de la camioneta US\$ 33 .
%\begin{equation}
%5hs*33\dfrac{US\$}{hs}=US\$ 165
%\end{equation}


\item El gasto en combustible:
Si consume 7,3 litros cada 100km, entonces en 66 km gastara 4,9 litros. El costo del combustible (diésel) por litro es de US \$ 0,66. El costo total de combustible es de US\$ 3,17.

\item Se tendrá un monto extra debido a peajes, compra de algún elemento o cualquier imprevisto que surja en el transcurso del plan de mantenimiento.  El monto extra será de US\$ 25
\end{enumerate}
%
El costo total del mantenimiento preventivo mensual es de US\$ 61,18 y anual de US\$ 734,16
%
\subsubsection{Mantenimiento Correctivo}

Se considera que habrá 8 fallas en la red fuera de la jornada laboral al año y 1 falla al mes durante la jornada laboral. 

Teniendo en cuenta que el técnico trabaja 2 horas fuera de la jornada laboral, el costo es:
\begin{itemize}
\item Vehículo: 8 días* US\$ 33=US\$ 264
\item Personal: 16 horas* US\$ 5=US\$ 80 
\item Combustible: 528km*0,66=US\$ 348,48
\end{itemize}

El costo total anual es de US\$ 692,48

El costo para fallas durante la jornada laboral está ya incluido cuando se realiza el mantenimiento predictivo.

\subsubsection{Nuevas Conexiones}
%
Las nuevas conexiones o desconexiones de los usuarios en la localidad serán cuando se realice mantenimiento preventivo y correctivo. 


\subsubsection{Costo Total}

Para el cálculo del costo total se tendrá en cuenta también los costos respecto a impuestos municipales y de servicios básicos de las localidades afectadas. Se considera un total de US\$300. 

El costo total del plan de mantenimiento al año es de \textbf{US\$ 1726,64}.















%---------------------------------------------------
%esta citacion fue hecha para probar que se puede
%citar a otro archivo

%\cite{base1}

%anduvo ok
%---------------------------------------------------




%Se procede a efectuar la inspección correspodiente de las conexiones, conectores y patchcords. Además debe realizarse la limpieza de los conectores SC y LC. A continuación se verifican los armacios distribuidores, las bandejas 
%
%Una vez finalizado se procede a rellenar los correspondientes documentos que se ilustran en la Tabla \ref{tab:formulario1} 


%\newpage

%------------------------------------------

\begin{thebibliography}{99}

%\bibitem{norma1}Norma UIT-T L.25 $"$Optical fibre cable network maintenance''

%\bibitem{norma2}Norma UIT-T M.20 "Mantenimiento: Introducción y principios generales''


\bibitem{costosherr}Costos de herrajes p/ montaje FO ADSS: \url{https://www.magusrl.com.ar/catalogo/listado/1//0/0/}

\bibitem{morsapk10}Morsa de suspensión pks-10: \begin{tiny}
\url{https://articulo.mercadolibre.com.ar/MLA-666598388-morsa-de-suspension-con-fusible-mecanico-pks-10cf-lct-_JM?quantity=1#position=1&type=item&tracking_id=600cf7ba-7f35-44c2-a15c-d11d0194210d
}
\end{tiny}

\bibitem{morsadr1500}Morsa DR1500: \url{https://www.magusrl.com.ar/catalogo/detalle/morsa-dr1500-autoajustable-5070-mm-con-cuna-chica
}

\bibitem{ojal}Ojal cerrado MN-380: \url{https://articulo.mercadolibre.com.ar/MLA-747727106-mn-380-ojal-cerrado-_JM?quantity=1
}

\bibitem{retencion}Retención preformada fibra optica: \url{https://articulo.mercadolibre.com.ar/MLA-752275208-retencion-preformada-para-fibra-optica-6874mm-amarillo-x5-_JM?quantity=1&variation=32075293484}

\bibitem{costo2tec} Costo Tecnico por hora: 
{\tiny \url{https://www.glassdoor.com.ar/Pago-mensual/Cablevision-Argentina-T\%C3\%A9cnico-Instalador-Pago-mensual-E828964_D_KO22,40.htm#:~:text=El\%20sueldo\%20t\%C3\%ADpico\%20de\%20un,variar\%20entre\%20\%2428.582\%20y\%20\%2482.332.}}


\bibitem{costo20} Costo Camioneta: 
\begin{tiny}
\url{https://www.urban-rentacar.com/buscar?dateFrom=07-07-202
0&dateTo=10-07-2020&dropOffEndpoint=Urban&dropOffId=1&hourFrom=9%3A00&hourTo=11%3
A00&ilimitedKm=false&pickUpEndpoint=Urban&pickUpId=1}
\end{tiny}





%\bibitem{costosherr}Costos de herrajes p/ montaje FO ADSS: \url{https://www.incom.mx/productos/A\%C3\%A9rea-AEREO1}

%\bibitem{catalogo}Catalogo Instalación Aerea \url{https://www.incom.mx/documents/pdf/CATALOGO_INCOM_AEREO.pdf}


\end{thebibliography}


\newpage

%------------------------------------------