En este capítulo se desarrolla el plan de acción en el cual brinda una planificación para mantener organizada la gestión y control de las tareas con el fin de cumplir con los objetivos del proyecto.

El plan de acción se analiza desde la compra a los proveedores hasta la terminación y funcionamiento de la Red, incluyendo el análisis de objetivos y actividades para alcanzarlo.

\section{Compra a Proveedores}
En la creación del plan de gestión de compras se identifica que los equipos que deben adquirirse son de proveedores nacionales y internacionales, por lo cual para cada uno la compra será diferente. En el análisis de vendedores se eligen los proveedores de mayor potencial, teniendo en cuenta la búsqueda de un compromiso entre los precios, calidad y tiempos para la llegada de productos y materiales.

Una vez seleccionados los proveedores y los procedimientos de compras,  se envían las solicitudes de compra. El plazo de entrega transcurre desde que se solicita el pedido hasta que llega al destino. Sin embargo algunos factores pueden afectar los plazos de entrega.

Los proveedores extranjeros entregan los equipos en un tiempo estimado de 60 días teniendo en cuenta la importación, aduana y transporte. 

Los proveedores nacionales entregan los equipos en un tiempo estimado de 15 días teniendo en cuenta el transporte.

En este proyecto se utilizan los postes de la Empresa EDESAL para el tendido de la fibra óptica, por lo cual se estima un periodo 60 días de negociación para el alquiler de los mismos.

Para el tendido de la red de fibra óptica, el Estado y las localidades implicadas deben dar su aprobación, por ende se provee un periodo de 30 días para las negociaciones a través de los organismos competentes.

Una vez que se ha completado el trabajo compra y negociaciones se debe documentar todos los procedimientos administrativos por medio de 2 empleados administrativos.

\section{Trabajos previos al tendido del cable}
Previo a la realización del tendido de cable de fibra óptica para la Red de Distribución y Red de Acceso se realizan las siguientes acciones:
\begin{itemize}
\item[•]\textbf{Señalización, acotación  y limpieza de las zonas de trabajo.}


\item[•]\textbf{Identificación de la ubicación de los postes sobre los cuales se va a desplegar la red.}

 
\end{itemize}
Estas acciones la realizan todas las cuadrillas de técnicos e Ingenieros que participan de la obra en un tiempo estimado de 2 días. 

El personal integrado por técnicos e ingenieros debe estar debidamente capacitado y certificado en instalación de F.O. Además deben tener los permisos adecuados si se trabaja cerca de líneas eléctricas de MT y AT.

\section{Puesta en marcha del enlace troncal}
La distancia a cubrir por la fibra óptica es de 30,2 Km, donde se realizan un gran número de trabajos. El tendido del cable es aéreo y tiene que ir precedido y seguido de diferentes tareas que completan la instalación:
\begin{itemize}

\item[•]El tendido.
\item[•]Empalme.
\item[•]Conexionado del cable.



\end{itemize}
El método que se implementa en la instalación es el siguiente:
\begin{enumerate}


\item[1]En primer lugar, se instala una serie de poleas temporales en cada poste a lo largo de la ruta.


\item[2]A continuación, la F.O. se desliza a través de las poleas y se fija en el extremo de la línea usando un fusible mecánico y un agarre de cable de tracción.



\FigureDEM{H}{10}{instalacion1}{Instalación de red FO. Aplicación de tensión.}


\item[3]La línea de tracción se utiliza para tirar del cable a través de apoyos de cable en su posición.

\item[4]La línea de cabrestante de tracción debe ser instalado a través de los soportes de cables. Una línea de cuerda o cabestrante no metálico debe ser utilizada para tirar del cable.

\item[5]Es vital que el cabrestante debe ser calibrado para detener la operación cuando la tensión de instalación excede la carga máxima del cable.

\item[6]Después de que el cable se ha tirado a su posición final, con holgura o reserva de cable para realizar el empalme o acceso, el cable debe tensarse hasta que se alcance el nivel de fecha correcto.

\item[7]A continuación, debe fijarse la ferretería definitiva en cada punto de apoyo a lo largo de la ruta.


\FigureDEM{H}{8}{instalacion2}{Instalación de red FO. Fijación de herrajes.}




\end{enumerate}

Para cubrir la distancia se estima un tiempo de 10 días, realizando 3 Km por día. El trabajo lo realizara una cuadrilla de técnicos (cuatro) y un ingeniero supervisor los cuales disponen de los equipos necesarios para la correcta ejecución de los trabajos.

Dentro del periodo total de la obra se construye una edificación de material (ladrillo, chapa, cemento, hierro, etc), una en Papagayos. En el interior de la edificación se colocan los equipos correspondientes. Esta casilla posee un cierre perimetral ajustándose a las disposiciones legales vigentes. Esta obra se realiza en un lapso de 10 días por 2 albañiles.

Se consideran 4 días más para la instalación y configuración de los equipos que se dispondrán en las ciudades de Cortaderas y Papagayos. La obra se realizara en un total de 14 días.

\section{Puesta en marcha de la Red de Acceso}
La distancia a cubrir por la fibra óptica es de 2,38 Km en forma de anillo. Dentro de la instalación de cable de fibra óptica se colocan 4 terminales de distribución. Cada terminal es un armario que en su interior hay splitter.

Se debe instalar y probar el anillo de fibra óptica, respetando el radio mínimo de curvatura especificado por el fabricante de la fibra. El cable de fibra óptica a utilizarse en el tendido deberá cumplir con las Normas  ANSI/EIA/TIA-568, EIA/TIA TSB-36, EIA/TIA TSB-40 y EIA/TIA SP-2840. Luego se verifican los terminales en las cajas de empalme de fibra óptica de los gabinetes de distribución.

El método de instalación es el mismo que se emplea en el enlace troncal. Para la realización del tendido de F.O. se estima 6 días y para las cajas de distribución con los splitters se estima 3 días, por lo cual el tiempo de obra es de 9 días.

La obra está realizada por una cuadrilla de técnicos (cuatro) y un ingeniero supervisor.

Se despliega el tendido del cable de acometida desde la terminal de distribución hasta la casa del cliente, ya sea de forma directa si se encuentra cerca de la terminal de conexión, o a través de postes.

Como existen 61 clientes potenciales, se deduce que aproximadamente se instalaran 12 clientes por día. La duración del tendido de la F.O. e instalación de equipos debe finalizar en 5 días.
\section{Trabajos luego de la terminación del tendido del cable}

\section{Diagrama de Gantt del Plan de Acción}

\section{Análisis económico del Plan de Acción}

\subsection{Costo de Mano de Obra}

\subsection{Costo de Medio de Transporte}

\subsection{Costo Total del Plan de Acción}