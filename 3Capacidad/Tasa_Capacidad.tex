\section{Planes y cálculo de capacidad}

%Los servicios que se desean brindar abarca toda la Localidad de Papagayos. En el análisis socio-económico se estiman que hay 170 hogares en esta localidad. De acuerdo con el análisis de la población se concluye brindarle el servicio al 80 \% de los hogares de dicha Localidad, los cuales, son potenciales clientes, lo cual, indica un total de 136 casas. 

Al plantear 55 casas abonadas al servicio, y considerando 6 nuevos abonados correspondientes a establecimientos que no son hogares, sino que son locales comerciales, clubes sociales y deportivos, entre otros, se calcula la capacidad máxima para el enlace junto a las diferentes tasas a ofrecer.

Se investiga que empresas brindan servicio de TICs, consultando por medio de las redes sociales y llamadas a contactos locales del Departamento de Chacabuco. 

Las empresas que brindan servicios de televisión en la región son:
\begin{itemize}
\item[•]DirecTV proveyendo más de 1.000 canales (HD y SD).
\item[•]Proveedores privados los cuales no brindan garantías respecto al servicio.
\end{itemize}

Se determina brindar 50 canales,  los cuales serán 40 canales SD y 10 canales HD. 

Mientras que las empresas que brindan servicios de telefonía: 

\begin{itemize}
\item[•]En telefonía fija: Cooperativa de Merlo y Cooperativa de Concaran.
\item[•]En telefonía móvil: la empresa Claro funciona en la periferia del pueblo y Movistar en menor medida. 
\end{itemize}



Las empresas que brindan servicios de Internet:
\begin{itemize}
\item[•]Cooperativa de Merlo con una velocidad de 4Mb.
\item[•]Antenas del gobiernos 2.0 y 3.0 .
\end{itemize}
La desventaja de estas últimas es que mirando el mapa de antenas de la localidad se observa que en horas pico tiene 70 conectados aproximadamente. Lo cual nos indica un nivel considerable de la ocupación, por lo que se sospecha que la calidad del servicio es limitada.

De acuerdo con lo investigado a nivel socio-económico de la población de la Localidad, se decide brindar un servicio de Internet el cual tendrá 3 opciones las cuales son: de hasta 2 Mbps, de hasta 4 Mbps y de hasta 6 Mbps teniendo en cuenta un factor de simultaneidad del 25 \%.  

Se decide brindar un servicio de VoIP, el cual tendrá una tasa de 30kbps y no se usa un factor de simultaneidad  debido a que su tasa es baja comprada con la de televisión e internet.

Cabe decir que se decide que los servicios a brindar tienen 3 niveles, los cuales se describen en la Tabla \ref{tab:niv-serv}.
%%---
% Table generated by Excel2LaTeX from sheet 'Hoja1'
\begin{table}[H]
  \centering
    \begin{tabular}{|l|c|c|c|c|}
    \hline
    \rowcolor[HTML]{C5D9F1} \textbf{Planes} & \multicolumn{1}{l|}{\textbf{VoIP}} & \multicolumn{1}{l|}{\textbf{SDTV}} & \multicolumn{1}{l|}{\textbf{HDTV}} & \multicolumn{1}{l|}{\textbf{Internet}} \bigstrut\\
    \hline
    \rowcolor[HTML]{C5D9F1} \textbf{Bronce} & \cellcolor[rgb]{ 1,  1,  1}Si & \cellcolor[rgb]{ 1,  1,  1}40 canales & \cellcolor[rgb]{ 1,  1,  1}- & \cellcolor[rgb]{ 1,  1,  1}2 Mbps \bigstrut\\
    \hline
    \rowcolor[HTML]{C5D9F1} \textbf{Plata} & \cellcolor[rgb]{ 1,  1,  1}Si & \cellcolor[rgb]{ 1,  1,  1}40 canales & \cellcolor[rgb]{ 1,  1,  1}5 canales & \cellcolor[rgb]{ 1,  1,  1}4 Mbps \bigstrut\\
    \hline
    \rowcolor[HTML]{C5D9F1} \textbf{Oro} & \cellcolor[rgb]{ 1,  1,  1}Si & \cellcolor[rgb]{ 1,  1,  1}40 canales & \cellcolor[rgb]{ 1,  1,  1}10 canales & \cellcolor[rgb]{ 1,  1,  1}6 Mbps \bigstrut\\
    \hline
    \end{tabular}%
  \caption{Niveles de servicios a brindar.}
  \label{tab:niv-serv}%

\end{table}%

%%---
%\FigureDEM{H}{9}{fig112}{Niveles de Servicios a brindar.}
%%--


Teniendo en cuenta los 61 clientes y los datos socio-económicos de la Localidad, se realiza una división aproximada de la cantidad de hogares por nivel de servicio. La columna de “Otros clientes” corresponde a candidatos como radios, escuela, restaurant, Municipalidad, etc.

%--
% Table generated by Excel2LaTeX from sheet 'Hoja1'
\begin{table}[H]
  \centering
    \begin{tabular}{|l|c|c|}
    \hline
    \rowcolor[HTML]{C5D9F1} \textbf{Plan} & \multicolumn{1}{l|}{\textbf{Casas Clientes}} & \multicolumn{1}{l|}{\textbf{Otros Clientes}} \bigstrut\\
    \hline
    \rowcolor[HTML]{C5D9F1} \textbf{Bronce} & \cellcolor[rgb]{ 1,  1,  1}29 & \cellcolor[rgb]{ 1,  1,  1}- \bigstrut\\
    \hline
    \rowcolor[HTML]{C5D9F1} \textbf{Plata} & \cellcolor[rgb]{ 1,  1,  1}19 & \cellcolor[rgb]{ 1,  1,  1}- \bigstrut\\
    \hline
    \rowcolor[HTML]{C5D9F1} \textbf{Oro} & \cellcolor[rgb]{ 1,  1,  1}7 & \cellcolor[rgb]{ 1,  1,  1}6 \bigstrut\\
    \hline
    \end{tabular}%
  \caption{Distribucion de numeros de clientes por servicio.}
  \label{tab:num-clientes}%
\end{table}%



%--
%\FigureDEM{H}{8}{Clientes}{Distribución de números de clientes por nivel de servicio.}
%--

% Tecnicamente estefania marco algo en esta frase, pero no entiendo que quiso decir, asi que no lo modifico

Para realizar el cálculo de la capacidad de enlace troncal se tendrá en cuenta la simultaneidad de cada servicio y las tasas de cada servicio que se muestran en la Tabla \ref{tab:factores-simultaneidad}.
%--
% Table generated by Excel2LaTeX from sheet 'Hoja1'
\begin{table}[H]
  \centering
    \begin{tabular}{|l|c|}
    \hline
    \rowcolor[HTML]{C5D9F1} \textbf{Servicio} & \multicolumn{1}{l|}{\textbf{Factor Simultaneidad}} \bigstrut\\
    \hline
    SDTV & 1:2 \bigstrut\\
    \hline
    HDTV & 1:1 \bigstrut\\
    \hline
    VoIP & 1:1 \bigstrut\\
    \hline
    Internet & 1:4 \bigstrut\\
    \hline
    \end{tabular}%
  \caption{Factores de simultaneidad de los servicios.}
  \label{tab:factores-simultaneidad}%
\end{table}%

%--
%\FigureDEM{H}{8}{fig52}{Factores de simultaneidad de los servicios.}
%--
Se comienza con el cálculo de la tasa de transmisión del enlace troncal, la cual varía de acuerdo al servicio analizado. Los métodos de cálculo se expresan a continuación:

\begin{itemize}

\item[•]Servicio VoIP e Internet: La taza de transmisión de estos servicios está completamente ligada al número de clientes que lo contrata. La forma de calcular para ambos casos la tasa de transmisión neta, consiste en multiplicar la tasa de transmisión propia de cada servicio por el número de clientes que los contrate. 

Para el caso de VoIP:
\begin{equation}
T_{NETA-SERVICIO}=T_{Servicio} * N_{Clientes} * f_{Simultaneidad}
\label{e1}
\end{equation}
Utilizando la Ecuación \ref{e1}:
\begin{equation}
T_{VoIP}=30Kbps * 61 * 1=1.83 Mbps 
\end{equation}

Para el caso de Internet, el cálculo de la tasa para cada uno de los planes ofrecidos se tiene incorporado el factor de simultaneidad.

\begin{itemize}
\item[•]Plan Bronce:
Utilizando la Ecuación \ref{e1} para este plan tenemos:
\begin{equation}
T_{INTERNET-BRONCE}= 2Mbps * N_{Clientes} * f_{Simultaneidad}=2Mbps * 29 * \frac{1}{4}=14.5Mbps
\end{equation}

\item[•]Plan Plata:
Utilizando la Ecuación \ref{e1} para este plan tenemos:
\begin{equation}
T_{INTERNET-PLATA}= 4Mbps * N_{Clientes} * f_{Simultaneidad}=4Mbps * 19 * \frac{1}{4}=19Mbps
\end{equation}

\item[•]Plan Oro:
Utilizando la Ecuación \ref{e1} para este plan tenemos:
\begin{equation}
T_{INTERNET-ORO}= 6Mbps * N_{Clientes} * f_{Simultaneidad}=6Mbps * 13 * \frac{1}{4}=19.5Mbps
\end{equation}
\end{itemize}

\item[•] Servicio de Televisión: La tasa de transmisión de este servicio es totalmente independiente del número de clientes que lo contrata. De acuerdo a la Tabla \ref{tab:niv-serv} el método para calcular la tasa neta consiste en multiplicar el número de canales almacenados en el caché del servidor con respecto a  la tasa de transmisión que corresponde.

Utilizando la Ecuación \ref{e2} para las distintas calidades de televisión tenemos:
\begin{equation}
T_{NETA-TV}=T_{Servicio}*N_{Canales Buffering}
\label{e2}
\end{equation}

Se decide almacenar 20 canales para el servicio SDTV, de manera que la tasa calculada es la siguiente:



\begin{equation}
T_{SDTV}=1.5Mbps*N_{Canales} =1.5Mbps * 20 =30Mbps
\end{equation}




Mientras que para el servicio HDTV se decide almacenar en la memoria caché todos los canales, de manera que la tasa calculada es la siguiente:





\begin{equation}
T_{HDTV}=6Mbps*N_{Canales}=6Mbps * 10 =60Mbps
\end{equation}

\end{itemize}

La contribución a la Tasa Neta de todos los servicios se calcula a continuación:
\begin{equation}
T_{NETA}=k(T_{VoIP}+T_{INTERNET-BRONCE}+T_{INTERNET-PLATA}+T_{INTERNET-ORO})+T_{SDTV}+T_{HDTV}=
\label{e3}
\end{equation}

Siendo k igual 1.5, la cual es una constante para establecer una capacidad remanente para posible crecimientos futuros en la misma localidad o hacia pueblos cercanos como Concaran, Cortaderas, Villa Larca,etc.

%Donde k es una constante que se le asigna un valor de 1,5, la cual, representa futuras ampliaciones de la red en la misma localidad o hacia pueblos cercanos como Concaran, Cortaderas, Villa Larca,etc.

Utilizando la Ecuación \ref{e3}:
\begin{equation}
T_{NETA}=1.5(1.83 Mbps +  14.5Mbps + 19Mbps + 19.5Mbps)+30Mbps+60Mbps=172.325 Mbps
\end{equation}





